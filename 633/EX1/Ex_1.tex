\documentclass[12pt]{article}

\usepackage[margin=1in]{geometry}
\usepackage{amsmath,amsthm,amssymb}
\usepackage{enumitem}
\usepackage{physics}

\newcommand{\magsq}[1]{\big|#1\big|^2}
\newcommand{\avg}[1]{\left<#1\right>}
\newcommand{\fullint}{\int_{-\infty}^\infty}
\newcommand{\fullintd}[1]{\fullint\dd#1\:}
\newcommand{\cint}[2]{\int_{#1}^{#2}}
\newcommand{\cintd}[3]{\cint{#1}{#2}\dd#3\:}

\begin{document}
	
\title{Exercise Set 1}
\author{Sean Ericson \\ Phys 633}
\maketitle

\section*{Exercise 1}
First let's right out all the probabilites we'll need:
\begin{align*}
    P(sick) &= x \\
    P(+|sick) &= 1-x \\
    P(-|sick) &= x \\
    P(+|\neg sick) &= x \\
    P(+) &= P(+|sick)P(sick) + P(+|\neg sick)P(\neg sick) = 2x(1-x) \\
    x &= 0.001
\end{align*}
Where the symbols + and - represent positive and negative test results (respectively), while $sick$ and $\neg sick$ represent being sick and healthy (respectively). Now the posterior probability of actually having Bayes' syndrom, given that you have tested positive, is simply
\[ P(sick|+) = \frac{P(+|sick)P(sick)}{P(+)} = \frac{x(1-x)}{2x(1-x)} = \frac{1}{2}. \]
In this case, the prior is the probability of any random person having Bayes' syndrome ($x=0.001$), the likelihood is the probability of a person with Bayes' syndrome testing positive for it ($1-x=0.999$), and the renormalization factor is the probability of \textit{any} test for Bayes' syndrome being positive.

\section*{Exercise 2}
The relevent probabilites are
\begin{align*}
    P(sick) &= 0.5 \\
    P(+|sick) &= 0.6 \\
    P(-|sick) &= 0.4 \\
    P(-|\neg sick) &= 1 \\
    P(-) &= P(-|sick)P(sick) + P(-|\neg sick)P(\neg sick) = 0.5(0.4 + 1) = 0.7
\end{align*}
The probability of being sick despite a negative test is therefore
\[ P(sick|-) = \frac{P(-|sick)P(sick)}{P(-)} = \frac{0.4*0.5}{0.7} = \frac{2}{7} \] 
In the case that the test's sensitivity is $90\%$, the likelihood and renormalization factor change to
\begin{align*}
    P(-|sick) &= 0.1 \\
    P(sick) = 0.5(0.1 + 1) &= .55
\end{align*}
and the posterior probability becomes
\[ P(sick|-) = \frac{0.1 * 0.5}{0.55} = \frac{1}{11} \]

\section*{Exercise 3}
\begin{enumerate}[label=(\alph*)]
    \item \[ B = A + (B - A) \implies \frac{1}{B}B\frac{1}{A} = \frac{1}{B}A\frac{1}{A} + \frac{1}{B}(B-A)\frac{1}{A} \implies \frac{1}{A} = \frac{1}{B} + \frac{1}{B}(B-A)\frac{1}{A} \]
    \item Let $A = z - H_0 - \lambda V$, $B = z - H_0$. Then, $B - A = \lambda V$ and
    \[ \frac{1}{z - H_0 - \lambda V} = \frac{1}{z-H_0} + \frac{1}{z-H_0}(\lambda V)\frac{1}{z-H_0-\lambda V}.  \]
    Because $G(z) = 1/A$ and $G_0(z) = 1/B$, this shows that
    \[ G(z) = G_0(z) + \lambda G_0(z)VG(z). \]
\end{enumerate}

\end{document}