\documentclass[12pt]{article}

\usepackage[margin=1in]{geometry}
\usepackage{amsmath,amsthm,amssymb}
\usepackage{mathrsfs}
\usepackage{mathtools}
\usepackage{enumitem}
\usepackage{physics}

\newcommand{\magsq}[1]{\big|#1\big|^2}
\newcommand{\avg}[1]{\left<#1\right>}
\newcommand{\fullint}{\int_{-\infty}^\infty}
\newcommand{\fullintd}[1]{\fullint\dd#1\:}
\newcommand{\cint}[2]{\int_{#1}^{#2}}
\newcommand{\cintd}[3]{\cint{#1}{#2}\dd#3\:}
\newcommand\treq{\stackrel{\mathclap{\tiny\mbox{Tr}}}{=}}

\begin{document}
	
\title{Exercise Set 4}
\author{Sean Ericson \\ Phys 633}
\maketitle

\section*{Monday}
\subsection*{Exercise 1}
The ``actual'' current distribution can be thought of as a linear superposition of classical circular trajectories. The linearity allows the logic to carry through.


\section*{Tuesday}
\subsection*{Exercise 1}
\begin{align*}
    \Delta E_\text{hfs}(\text{1S},F) & = -\frac{2\pi g_I g_S \alpha^2 a_0}{3m_e}\avg{\vec{S}\cdot\vec{I}}\magsq{\psi_{100}(\vec{0})} \\
    &= -\frac{2\pi g_I g_S \alpha^2 a_0}{3m_e}\left(\frac{\hbar^2}{2}F(F+1)-\frac{3}{4}\right)\left(\frac{1}{\pi a_0^3}\right) \\
    &= -\frac{g_I g_S \alpha^2 \hbar^2}{6m_e a_0^2} \begin{cases}
        \times 1 & \text{if $F = 1$ (triplet)} \\
        \times (-3) & \text{if $F = 0$ (singlet)}
      \end{cases}
\end{align*}


\end{document}