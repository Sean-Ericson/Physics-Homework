\documentclass[12pt]{article}

\usepackage[margin=1in]{geometry}
\usepackage{amsmath,amsthm,amssymb}
\usepackage{mathtools}
\usepackage{mathrsfs}
\usepackage{enumitem}
\usepackage{physics}
\usepackage{empheq}
\usepackage{undertilde}
\usepackage{float}
\usepackage{graphicx}
\graphicspath{ {./images/} }

\usepackage{tikz}
\usetikzlibrary{calc,decorations.markings,patterns}

\newcommand{\magsq}[1]{\big|#1\big|^2}
\newcommand{\avg}[1]{\left<#1\right>}
\newcommand{\fullint}{\int_{-\infty}^\infty}
\newcommand{\fullintd}[1]{\fullint\dd#1\:}
\newcommand{\cint}[2]{\int_{#1}^{#2}}
\newcommand{\cintd}[3]{\cint{#1}{#2}\dd#3\:}

\begin{document}

\title{Problem Set \#1}
\author{Phys 614}
\date{Due: 12 Noon, Thursday, April 13, 2023}
\maketitle

\section*{Problem 1}
For a quantum-mechanical system of non-interacting particles in the Grand Canonical Ensemble, calculate the probability $P(N)$ that the system has \underline{exactly} $N$ particles. Express your answer \underline{entirely} in terms of $N$ and the mean number of particles $\bar{N}$.

\section*{Problem 2}
Do the same as (1) for a system of \underline{classical}, but indistinguishable, non-interacting particles all moving in a common potential $V(\vec{r})$.

\section*{Problem 3}
What is the statistical significance of your answers to problems (1) and (2)?

\section*{Problem 4}
Consider a system of non-interacting ``$N$-yons'', particles which are between Fermions and Bosons in the following sense: For ``$N$-yons'', the largest number of particles allowed in one quantum-mechanical state is $N_{\text{max}}$ ($1 \leq N_{\text{max}} \leq \infty$), independent of the state.
\begin{enumerate}[label=(\alph*)]
    \item Find the mean occupation number $\ev{n_i}$ of a state with single-particle energy $\epsilon_i$.
    \item Show that you recover the Fermi result for $\ev{n_i}$ if $N_{\text{max}}=1$.
    \item Show that you recover the Bose result for $N_{\text{max}}\to\infty$.
\end{enumerate}

\end{document}