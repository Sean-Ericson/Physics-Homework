\documentclass[12pt]{article}

\usepackage[margin=1in]{geometry}
\usepackage{amsmath,amsthm,amssymb}
\usepackage{mathtools}
\usepackage{mathrsfs}
\usepackage{enumitem}
\usepackage{physics}
\usepackage{empheq}
\usepackage{undertilde}
\usepackage{float}
\usepackage{graphicx}
\graphicspath{ {./images/} }

\usepackage{tikz}
\usetikzlibrary{calc,decorations.markings,patterns}

\newcommand{\magsq}[1]{\big|#1\big|^2}
\newcommand{\avg}[1]{\left<#1\right>}
\newcommand{\fullint}{\int_{-\infty}^\infty}
\newcommand{\fullintd}[1]{\fullint\dd#1\:}
\newcommand{\cint}[2]{\int_{#1}^{#2}}
\newcommand{\cintd}[3]{\cint{#1}{#2}\dd#3\:}

\begin{document}

\title{Problem Set \#2}
\author{Phys 614}
\date{Due: Who the hell knows when}
\maketitle

\section*{Problem 1}
$N$ non-interacting Bosons of mass m move in a spherically symmetric potential (in three dimensions)
\begin{equation}
    U(\vec{r}) = \frac{1}{2}m\omega r^2 \qquad ( r \equiv \abs{\vec{r}})
\end{equation}
\begin{enumerate}[label=\Alph*)]
    \item Calculate the Bose-Einstein condensation temerature $T_c$ of this system.
    \item Calculate the number of atoms in the ground state as a function of temperature $T < T_c$. \\ (Note: this is a reasonable model for the famous Bose-Einstein condensation experiments of Cornell and Weinmann, who created the potential with a ``magnetic trap'')
\end{enumerate}

\section*{Problem 2}
Using the result of (1), plus your knowledge of the ground state wavefunction for a harmonic oscillator, calculate the density of atoms at the origin for $T < T_c$, and show that it is larger than the corresponding density for $T \gtrapprox T_c$ (e.g. $T = 2T_c$) by a factor of order $\sqrt{N}$, which, of course, is huge. Comment on the implications of this result for the applicability of the non-interacting approximation in this experiment.

\section*{Problem 3}
Returning to Bose-Einstein condensation of particles in a box (rather than a harmonic potential), show that there is no Bose-Einstein condensation in $d=2$ dimensions of space. \\ (Hint: calculate $\rho(T, \mu)$ by replacing sums by integrals in the usual way, and show that $\rho$ is unbounded (i.e. diverges) as $\mu \to 0^-$ from below.)



\end{document}