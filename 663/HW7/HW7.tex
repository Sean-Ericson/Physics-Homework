\documentclass[12pt]{article}

\usepackage[margin=1in]{geometry}
\usepackage{amsmath,amsthm,amssymb}
\usepackage{nccmath}
\usepackage{mathtools}
\usepackage{mathrsfs}
\usepackage{enumitem}
\usepackage{physics}
\usepackage{slashed}
\usepackage{pdfpages}
\usepackage{float}

\usepackage{tikz}
\usepackage{tikz-feynman}

\newcommand{\magsq}[1]{\big|#1\big|^2}

\begin{document}

\title{Homework 6}
\author{Sean Ericson \\ Phys 663}
\maketitle

\section*{Problem 1}
\begin{enumerate}[label=\roman*)]
    \item For a radiation dominated flat universe, we have that
    \begin{alignat*}{3}
        & \quad & a(t) &= \left(\frac{t}{t_0}\right)^{1/2}  \\
        &\implies\quad & t &= t_0 a^2 \\
        &\implies\quad & \dd t &= 2t_0 a\dd a. \\
    \end{alignat*}
    Then,
    \begin{alignat*}{3}
        & \quad & H(t) &= \frac{1}{2}t^{-1}  \\
        &\implies\quad & H_0 &= \frac{1}{2}t_0^{-1} \\
        &\implies\quad & t_0 &= \frac{1}{2}H_0^{-1}, \\
    \end{alignat*}
    and
    \begin{align*}
        d_\text{max}(t) &= a(t)\int_{0}^{t_0} \frac{\dd t'}{a(t')} \\
        &= 2t_0a(t)\int_0^1\dd a \\
        &= 2t_0a(t) \\
        &= H_0^{-1}a(t).
    \end{align*}
    Evaluating this at the current epoch gives a cosmological horizon of 
    \[ \boxed{d_\text{max}(t_0) = H_0^{-1}} \]

    \item For a matter dominated flat universe, we have that
    \begin{alignat*}{3}
        & \quad & a(t) &= \left(\frac{t}{t_0}\right)^{2/3}  \\
        &\implies\quad & t &= t_0 a^{3/2} \\
        &\implies\quad & \dd t &= \frac{3}{2} t_0 a^{1/2}\dd a. \\
    \end{alignat*}
    Then,
    \begin{alignat*}{3}
        & \quad & H(t) &= \frac{2}{3}t^{-1}  \\
        &\implies\quad & H_0 &= \frac{2}{3}t_0^{-1} \\
        &\implies\quad & t_0 &= \frac{2}{3}H_0^{-1}, \\
    \end{alignat*}
    and
    \begin{align*}
        d_\text{max}(t) &= a(t)\int_{0}^{t_0} \frac{\dd t'}{a(t')} \\
        &= \frac{3}{2}t_0\int_0^1\frac{\dd a}{a^{1/2}} \\
        &= 3t_0 \\
        &= 2H_0^{-1}.
    \end{align*}
    Evaluating this at the current epoch gives a cosmological horizon of 
    \[ \boxed{d_\text{max}(t_0) = 2H_0^{-1}} \]

    \item For a cosmological constant dominated flat universe, we have that
    \begin{alignat*}{3}
        & \quad & a(t) &= e^{H_0t}  \\
        &\implies\quad & t &= H_0^{-1}\ln(a) \\
        &\implies\quad & \dd t &= \frac{\dd a}{H_0 a}. \\
    \end{alignat*}
    Clearly there is no finite value of the time coordinate $t$ such that $a(t) = 0$. Spencer said to take $t_\text{beg} = 0$ in this case, but I'll just leave it as $t_\text{beg}$
    \begin{align*}
        d_\text{max}(t) &= a(t)\int_{t_\text{beg}}^{t_0} \frac{\dd t'}{a(t')} \\
        &= a(t) \int_{t_\text{beg}}^{t_0} \dd t' e^{-H_0 t'} \\
        &= -a(t) H_0^{-1} \left(a^{-1}(t_0)-a^{-1}(t_\text{beg})\right)
    \end{align*}
    Evaluating this at the current epoch gives a cosmological horizon of 
    \[ \boxed{d_\text{max}(t_0) = H_0^{-1}\left(a^{-1}(t_\text{beg}) - 1\right)} \]
\end{enumerate}

\section*{Problem 2}
Similarly to part ii) of Problem 1,
\begin{alignat*}{3}
    & \quad & d(t_0) &= \int_{t_\text{emit}}^{t_0}\frac{\dd t'}{a(t')} \\
    & \quad & &= H_0^{-1}\int_{a(t_\text{emit})}^1 \frac{\dd a}{a^{1/2}} \\
    & \quad & &= H_0^{-1}\int_0^z \frac{\dd z}{(1 + z)^{(3/2)}} \\
    & \quad & &= 2H_0^{-1}\left(1 - (1 + z)^{-1/2}\right) \\
    &\implies\quad & H_0d &= z - \frac{3}{4}z^2 + \frac{5}{8}z^3 + \cdots, \\
\end{alignat*}
so it looks like the quadratic coefficient is $-3/4$.

\section*{Problem 3}
\begin{enumerate}[label=(\alph*)]
    \item A static solution should have $\dot{a} = \ddot{a} = 0$. For the second Friedman equation this implies
    \[ \sum_i\left(\rho_i + 3p_i\right) = \sum_i(1 + 3w_i)\rho_i = 0. \]
    For normal matter $w = 0$, so we can write the above condition as
    \[ \rho_\text{matter} + (1 + 3w_\text{other})\rho_\text{other} = 0. \]
    Assuming positive energy densities, this implies
    \[ w_\text{other} = -\frac{1}{3}\left(\frac{\rho_\text{matter}}{\rho_\text{other}} + 1\right) < -\frac{1}{3} \]

    \item Let $\zeta = 8\pi G/3$, and assume positive energy densities. Then
    \[ \left(\frac{\dot{a}}{a}\right)^2 = 0 = \zeta\sum_i\rho_i - \frac{k}{a^2} \]
    \[ \implies k = \zeta a^2 \sum_i\rho_i > 0 \]

    \item We can rearrange the Friedman equation as
    \[ \dot{a}^2 - \zeta a^2 \sum_i\rho_i + k = 0, \]
    where as in class we can interpret the first term as a ``kinetic energy'' and the remaining terms as a ``potential energy''. The potential energy term is an inverted parabola, so clearly the equilibrium is unstable.
    
\end{enumerate}


\end{document}