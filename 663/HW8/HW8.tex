\documentclass[12pt]{article}

\usepackage[margin=1in]{geometry}
\usepackage{amsmath,amsthm,amssymb}
\usepackage{nccmath}
\usepackage{mathtools}
\usepackage{mathrsfs}
\usepackage{enumitem}
\usepackage{physics}

\newcommand{\magsq}[1]{\big|#1\big|^2}

\begin{document}

\title{Homework 8}
\author{Sean Ericson \\ Phys 663}
\maketitle

\section*{Problem 1}
\[ \Gamma = cT^n; \quad N_\text{int} = \int_t^\infty \dd t' \Gamma(t') \]
For a radiation dominated universe
\[ a(t) = \sqrt{\frac{t}{t_0}} \implies H = \frac{1}{2t}, \;\; H_0 = \frac{1}{2t_0} \]
\begin{alignat*}{3}
    & \quad & T &\propto \frac{1}{a} \\
    &\implies\quad & \Gamma &= c\left(\frac{t}{t_0}\right)^{n/2} \\
    &\implies\quad & N_\text{int} &= c\int_t^\infty \dd t' \left(\frac{t}{t_0}\right)^{n/2} \\
    & \quad &   &= \frac{2c}{n-2} \frac{t^{-(n-2)/2}}{(2H_0)^{n/2}} \quad (n > 2) \\
\end{alignat*}
Now,
\begin{alignat*}{3}
    & \quad & N_\text{int}(t_d) &= 1 \\
    &\implies\quad & \frac{2c}{n-2} \frac{t_d^{-(n-2)/2}}{(2H_0)^{n/2}} &= 1 \\
    &\implies\quad & t_d &= \frac{1}{2}\left(\frac{n-2}{c}\right)^{\frac{-2}{n-2}}H_0^{\frac{-n}{n-2}}.
\end{alignat*}
Finally,
\begin{alignat*}{3}
    & \quad & \frac{\Gamma(t)}{H(t)} &= c \frac{t^{(2-n)/2}}{t_0^{-n/2}} \\
    &\implies\quad & \frac{\Gamma(t_d)}{H(t_d)} = \frac{n-2}{2},
\end{alignat*}
which is greater than 1 for $n>4$

\section*{Problem 2}
\begin{enumerate}[label=\roman*)]
    \item The entropy density is given by
    \begin{align*}
        s_0 &= \frac{2\pi^2}{45}g_{*s}T_\gamma^3 \\
        &= \frac{2\pi^2}{45}\left(2 + \frac{7}{8}\times3\times2\left(\frac{T_\nu}{T_\gamma}\right)^3\right)T_\gamma^3 \\
        &= \frac{2\pi^2}{45}\left(2 + \frac{21}{4}\frac{4}{11}\right)(2.73\;\text{K})^3 \\
        &= 39.4\;\text{K}^3 \\
        &\approx 4\times10^{-38}\;\text{GeV}^3
    \end{align*}
    The critical density is
    \[ \rho_c = \frac{3H_0^2}{8\pi G} = 5\times10^{-6}\;\text{GeV}\text{cm}^{-3} \]
    the dark matter density is then
    \[ \rho_\text{DM} = \Omega_\text{DM}\rho_c \approx 1.3\times10^{-6}\text{GeV}\text{cm}^{-3} \]
    The number density, $n_\text{DM}$ is given by
    \[ n_\text{DM} = Y_\text{DM} s_0. \]
    Using 
    \[ Y_\text{DM} \sim 0.2 \frac{g}{g_{*s}} \approx 0.007 \]
    I get
    \[ n_\text{DM} \approx 3\times10^{-40}\;\text{GeV}^3 \]
    \item 
    \[ \Omega_\text{DM} \approx 0.25 = \frac{\rho_\text{DM}}{\rho_c} = \frac{m_\text{DM}n_\text{DM}}{\rho_c} \]
    \[ \implies m_\text{DM} = \frac{\Omega_\text{DM}\rho_c}{n_\text{DM}} \approx 37\;\text{eV} \]
    This is wayyyy below the weak scale. But I'm sure my yield calculation was nonsense =P.
\end{enumerate}


\section*{Problem 3}
Given that
\begin{align*}
    \rho_\text{DM} &= 0.3\;\text{GeV}\text{cm}^{-3}, \\
    R &= 20\;\text{kpc}, \\
    \left<\sigma v\right> &= 3\times10^{-26}\;\text{cm}^3\text{s}^{-1}, \\
    m_\text{DM} &= 100\;\text{GeV}
\end{align*}
we have that
\begin{alignat*}{3}
    & \quad & \Gamma &\approx n \left<\sigma v\right> \\
    & \quad &        &= \frac{\rho_\text{DM}}{m_\text{DM}}\left<\sigma v\right> \\
    & \quad &        &\approx  1\times10^{-28}\;\text{s}^{-1}
\end{alignat*}
The total number of dark matter particles within the given radius is
\[ N = \frac{4}{3}\pi R^3 \frac{\rho_\text{DM}}{m_\text{DM}} \approx 10^{66} \]
The current rate of annihilations in the galaxy is then $\sim10^{38}$/sec. Given that the time constant ($1/\Gamma$) is about 10 orders of magnitude larger than the current age of the universe, it seems we are not presently at great risk of galactic dark matter depletion.
Using
\[ \rho_\text{DM} = \Omega_\text{DM} \rho_c = \frac{3\Omega_\text{DM}H_0^2}{8\pi G} \approx 1.2\times10^{-6}\;\text{GeV}\text{cm}^{-3} \]
as the universal dark matter density, we find a universal dark matter annihilation rate of
\[ \Gamma \approx 10^{-34}\;\text{s}^{-1} \]

\end{document}