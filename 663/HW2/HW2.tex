\documentclass[12pt]{article}

\usepackage[margin=1in]{geometry}
\usepackage{amsmath,amsthm,amssymb}
\usepackage{nccmath}
\usepackage{mathtools}
\usepackage{mathrsfs}
\usepackage{enumitem}
\usepackage{physics}
\usepackage{slashed}
\usepackage{pdfpages}

\usepackage{tikz}
\usetikzlibrary{calc,decorations.markings,patterns}
\usetikzlibrary{decorations.pathmorphing}
\tikzset{snake it/.style={decorate, decoration=snake}}

\begin{document}

\title{Homework 2}
\author{Sean Ericson \\ Phys 662}
\maketitle
\section*{Problem 1 (Peskin 17.2)}
\begin{enumerate}[label=(\alph*)]
    \item From chapter 8,
    \[ \dv{\sigma}{\cos\theta} = \frac{\pi\alpha^2}{2s}\left(1\pm\cos\theta\right)^2, \]
    where the + is for the $RL\to RL$ and $LR\to LR$ helicity states, and the minus is for the $LR\to RL$ and $RL\to LR$ states.
    The recall is \textit{total}.
    \item
    \begin{figure}[h]
        \centering
        \resizebox{\textwidth}{!}{
            \begin{tikzpicture}[scale=1, every node/.style={scale=1.5}]
                % e +/-
                \draw[thick, decoration={markings,mark=at position 0.5 with {\arrow{>}}},postaction={decorate}] (0,8) node[left]{$e^-$} -- (3,4);
                \draw[thick, decoration={markings,mark=at position 0.5 with {\arrow{>}}},postaction={decorate}] (3,4) -- (0,0) node[left]{$e^+$};
    
                % photon
                \draw[thick, snake it] (3,4)--(7.5,4) node[midway,below]{\huge$\gamma$};
    
                % mu +/-
                \draw[thick, decoration={markings,mark=at position 0.5 with {\arrow{>}}},postaction={decorate}] (7.5,4)--(10.5,8) node[right]{$\mu^-$};
                \draw[thick, decoration={markings,mark=at position 0.5 with {\arrow{>}}},postaction={decorate}] (7.5,4)--(10.5,0) node[right]{$\mu^+$};
    
                % Momentum
                \draw[->] (0.4,7) -- (1.5,5.5) node[midway,left]{$p_-$};
                \draw[->] (1,2) -- (2,3.25) node[midway,left]{$p_+$};
                \draw[->] (4.5,4.5) -- (6.5,4.5) node[midway,above]{$q$};
                \draw[->] (9,5.5) -- (10,6.75) node[midway,right]{$p'_-$};
                \draw[->] (8.5,3.25) -- (9.5,2) node[midway,right]{$p'_+$};
            \end{tikzpicture}
            \begin{tikzpicture}[scale=1, every node/.style={scale=1.5}]
                % e +/-
                \draw[thick, decoration={markings,mark=at position 0.5 with {\arrow{>}}},postaction={decorate}] (0,8) node[left]{$e^-$} -- (3,4);
                \draw[thick, decoration={markings,mark=at position 0.5 with {\arrow{>}}},postaction={decorate}] (3,4) -- (0,0) node[left]{$e^+$};
    
                % Z
                \draw[thick, snake it] (3,4)--(7.5,4);
                \draw[thick, snake it] (3,4)--(7.5,4) node[midway,below]{\huge$Z$};
    
                % mu +/-
                \draw[thick, decoration={markings,mark=at position 0.5 with {\arrow{>}}},postaction={decorate}] (7.5,4)--(10.5,8) node[right]{$\mu^-$};
                \draw[thick, decoration={markings,mark=at position 0.5 with {\arrow{>}}},postaction={decorate}] (7.5,4)--(10.5,0) node[right]{$\mu^+$};
    
                % Momentum
                \draw[->] (0.4,7) -- (1.5,5.5) node[midway,left]{$p_-$};
                \draw[->] (1,2) -- (2,3.25) node[midway,left]{$p_+$};
                \draw[->] (4.5,4.5) -- (6.5,4.5) node[midway,above]{$q$};
                \draw[->] (9,5.5) -- (10,6.75) node[midway,right]{$p'_-$};
                \draw[->] (8.5,3.25) -- (9.5,2) node[midway,right]{$p'_+$};
            \end{tikzpicture}
        }
        \caption{The Feynman diagram for $e^+e^- \to \mu^+\mu^-$}
        \label{fig1}
    \end{figure}
    Clearly, both diagrams are nearly identical. The only differences are the massiveness of the $Z$ boson, and the couplings of the $Z$ to $e$ and $\mu$. The virtual photon matrix element is 
    \[ \mathcal{M}_\gamma(e_L^-e_R^+\to\mu_L^-\mu_R^+) = e^2(1 + \cos\theta) = g^2s_w^2(1 + \cos\theta), \]
    while the virtual $Z$ matrix element is
    \begin{align*}
        \mathcal{M}_Z(e_L^-e_R^+\to\mu_L^-\mu_R^+) &= \frac{g^2}{c_w^2}\frac{Q_{ZL}^2 8E^2}{s - m_Z^2 + im_Z\Gamma_Z} \frac{1}{2}(1 + \cos\theta) \\
        &= \frac{g^2}{c_w^2}\frac{(\frac{1}{2} - s_w^2)^2 s}{s - m_Z^2 + im_Z\Gamma_Z}(1 + \cos\theta).
    \end{align*}
    The total matrix element is then
    \begin{align*}
        \mathcal{M}(e_L^-e_R^+\to\mu_L^-\mu_R^+) &= \mathcal{M_\gamma} + \mathcal{M_Z} \\
        &= g^2s_w^2(1 + \cos\theta) + \frac{g^2}{c_w^2}\frac{(\frac{1}{2} - s_w^2)^2 s}{s - m_Z^2 + im_Z^2}(1 + \cos\theta) \\
        &= g^2s_w^2(1 + \cos\theta)\left(1 + \frac{1}{c_w^2s_w^2}(\frac{1}{2}-s_w^2)^2\frac{s}{s-m_Z^2+im_Z\Gamma_Z}\right)
    \end{align*}
    The matrix element has the same $\theta$ dependence, so the cross section will be the same up to the term in large parenthesis above (squared).
    \item Plugging in the other charges / polarization product, we get
    \begin{align*}
        \mathcal{M}(e_R^-e_L^+\to\mu_R^-\mu_L^+) &= g^2s_w^2(1 + \cos\theta)\left(1 + \frac{s_w^2}{c_w^2}\frac{s}{s-m_Z^2+im_Z\Gamma_Z^2}\right)\\
        \mathcal{M}(e_L^-e_R^+\to\mu_R^-\mu_L^+) &= g^2s_w^2(1 - \cos\theta)\left(1 + \frac{1}{c_w^2}(-\frac{1}{2}+s_w^2)\frac{s}{s-m_Z^2+im_Z\Gamma_Z}\right)\\
        \mathcal{M}(e_R^-e_L^+\to\mu_L^-\mu_R^+) &= g^2s_w^2(1 - \cos\theta)\left(1 + \frac{1}{c_w^2}(-\frac{1}{2}+s_w^2)\frac{s}{s-m_Z^2+im_Z\Gamma_Z}\right)
    \end{align*}
    \item
    \begin{align*}
        \sigma(\cos\theta > 0) &= \int_0^1\dd \cos\theta \left(1 + \cos\theta\right)^2 \\
        &= \frac{1}{3}\left(1 + \cos\theta\right)^3\big|_{\cos\theta=0}^1 \\
        &= \frac{7}{3} \\
        \sigma(\cos\theta < 0) &= \int_{-1}^0\dd \cos\theta \left(1 + \cos\theta\right)^2 \\
        &= \frac{1}{3}\left(1 + \cos\theta\right)^3\big|_{\cos\theta=-1}^0 \\
        &= \frac{1}{3} 
    \end{align*}
    \[ \implies  \frac{\sigma(\cos\theta > 0) - \sigma(\cos\theta < 0)}{\sigma(\cos\theta > 0) + \sigma(\cos\theta < 0)} = \frac{7 - 1}{7+1} = \frac{3}{4} \]
    \item In the limit $s \ll m_Z^2$, the breit-wigner factors go to zero (to leading order), reducing the cross sections to the ones calculated in chapter 8. The $LR\to LR$ and $RL\to RL$ states have $A_{FB} = 3/4$, while the other two states have $A_{FB} = -3/4$\footnote{I did the integrals; they're practically identical to the ones in part (d), do I really need to type them out?}. Clearly, then, the unpolarized process has vanishing forward-backward asymmetry in this limit. \\
    When $s = m_Z^2$, the Breit-wigner factors reduce to $-im_Z/\Gamma_Z$. Neglecting the photon contribution, the cross sections are
    \begin{align*}
        \dv{\sigma}{\cos\theta}\left(e_L^-e_R^+\to\mu_L^-\mu_R^+\right) &= \left[g^2s_w^2(1 + \cos\theta)\frac{1}{c_w^2s_w^2}(\frac{1}{2}-s_w^2)^2\frac{m_Z}{\Gamma_Z} \right]^2 \\
        \dv{\sigma}{\cos\theta}\left(e_R^-e_L^+\to\mu_R^-\mu_L^+\right) &= \left[g^2s_w^2(1 + \cos\theta)\frac{s_w^2}{c_w^2}\frac{m_Z}{\Gamma_Z} \right]^2 \\
        \dv{\sigma}{\cos\theta}\left(e_L^-e_R^+\to\mu_R^-\mu_L^+\right) &= \left[g^2s_w^2(1 - \cos\theta)\frac{1}{c_w^2}(-\frac{1}{2}+s_w^2)\frac{m_Z}{\Gamma_Z} \right]^2 \\
        \dv{\sigma}{\cos\theta}\left(e_R^-e_L^+\to\mu_L^-\mu_R^+\right) &= \left[g^2s_w^2(1 - \cos\theta)\frac{1}{c_w^2}(-\frac{1}{2}+s_w^2)\frac{m_Z}{\Gamma_Z} \right]^2
    \end{align*}
    Let's preemptively drop common terms that will end up canceling. The total cross section is
    \[ \dv{\sigma}{\cos\theta} \propto (f_1 + f_2)(1  + \cos\theta)^2 + 2f_3(1-\cos\theta)^2 \]
    where
    \begin{align*}
        f_1 &= \frac{\left(\frac{1}{2} - s_w^2\right)^4}{s_w^4} \\
        f_2 &= s_w^4 \\
        f_3 &= \left(\frac{1}{2} - s_w\right)^2
    \end{align*}
    The forward and backward integrals give
    \begin{align*}
        \sigma_{>0} &= \frac{7}{3}(f_1 + f_2) - \frac{2}{3}f_3 \\
        \sigma_{<0} &= \frac{1}{3}(f_1 + f_2) - \frac{14}{3}f_3
    \end{align*}
    Hence,
    \begin{align*}
        A_{FB} &= \frac{6(f_1 + f_2) + 12f_3}{8(f_1 + f_2)-16f_3}
    \end{align*}
    \item In the ultra-high energy limit, the Breit-Wigner factor goes to 1, and we get
    \[ \dv{\sigma}{\cos\theta}\left(e_L^-e_R^+\to\mu_L^-\mu_R^+\right) = \left[g^2s_w^2(1 + \cos\theta)\left(1 + \frac{1}{c_w^2s_w^2}(\frac{1}{2}-s_w^2)^2\right)\right]^2 \]
    \item The $B$ diagram gives 
    \[ (-\frac{1}{2}g')^2(1 + \cos\theta) \]
    The $A$ diagram gives 
    \[ (-\frac{1}{2}g)^2(1 + \cos\theta) \]
    I tried takeing all the $s_w$ and $c_w$s in (f) and turning them into $g$s and $g'$s, but I can't get them to look equal =(.
    \item For $RL\to RL$ The $A$ diagram doesn't contribute, but the $B$ diagram gives
    \[ (-g')^2(1 + \cos\theta) \]
    For $LR\to RL$ and $RL \to LR$ The $A$ diagram also doesn't contribute, but $B$ gives
    \[ (\frac{1}{2}g')^2(1 - \cos\theta) \] 
\end{enumerate}

\end{document}