\documentclass[12pt]{article}

\usepackage[margin=1in]{geometry}
\usepackage{amsmath,amsthm,amssymb}
\usepackage{nccmath}
\usepackage{mathtools}
\usepackage{mathrsfs}
\usepackage{enumitem}
\usepackage{physics}
\usepackage{slashed}
\usepackage{pdfpages}
\usepackage{float}

\usepackage{tikz}
\usepackage{tikz-feynman}

\newcommand{\magsq}[1]{\big|#1\big|^2}

\graphicspath{ {./images/} }

\begin{document}

\title{Homework 5}
\author{Sean Ericson \\ Phys 663}
\maketitle

\section*{Problem 1}
\begin{figure}[H]
    \centering
    \resizebox{0.35\textwidth}{!}{
        \feynmandiagram [horizontal=i1 to p1] {
        i1[particle=$h$] -- [scalar, momentum'=$p$] p1[label=$\kappa$] -- [scalar, momentum'=$p_1$] f1[particle=$X$],
        p1 -- [scalar, momentum'=$p_2$] f2[particle=$X$],
        };
    }
    \caption{The decay $h \to X+X$.}
    \label{fig1}
\end{figure}
\begin{enumerate}[label=(\roman*)]
    \item The partial width of the decay is given by
    \[ \Gamma = \frac{1}{2m_h}\int \frac{\dd\Pi_2}{2!} \magsq{\mathcal{M}(h\to XX)}. \]
    The matrix element is $\kappa$, so we simply need to integrate over 2-body phase space. In the CM frame, 
    \[ p = (m_h, 0, 0, 0), \quad p_1 = (m_h/2, \abs{\vec{p_1}}\sin\theta, 0, \abs{\vec{p_1}}\cos\theta), \quad p_2 = (m_h/2, -\abs{\vec{p_1}}\sin\theta, 0, -\abs{\vec{p_1}}\cos\theta) \]
    \[ E_1^2 = m_X^2 + \magsq{\vec{p}_1} \]
    The phase space integral is then
    \begin{align*}
        \int\dd\Pi_2 &= \int \frac{\dd^3\vec{p}_1}{(2\pi)^32E_1}\frac{\dd^3\vec{p}_2}{(2\pi)^32E_2}(2\pi)^4\delta^4(p - p_1 - p_2) \\
        &= \frac{1}{4(2\pi)^2}\int\dd^3\vec{p}_1\dd^3\vec{p}_2\frac{1}{E_1E_2}\delta(E-E_1-E_2)\delta^3(\vec{p}_1+\vec{p}_2) \\
        &= \frac{1}{(2\pi)^2}\int\dd\abs{\vec{p}_1}\dd\Omega\frac{\magsq{\vec{p}_1}}{E_1^2}\delta(E-2E_1) \\
        &= \frac{4\pi}{4(2\pi)^2}\int\dd\abs{\vec{p}_1}\frac{\magsq{\vec{p}_1}}{E_1^2}\delta(E-2E_1) \\
        &= \frac{1}{4\pi}\frac{\abs{\vec{p}_1}^2}{E_1^2}\frac{E_1}{2\abs{\vec{p}_1}}\Biggr\rvert_{E_1=\frac{m_h}{2}} \\
        &= \frac{1}{8\pi}\frac{\abs{\vec{p}_1}}{E_1}\Biggr\rvert_{E_1=\frac{m_h}{2}} \\
        &= \frac{1}{4\pi m_h}\sqrt{\left(\frac{m_h}{2}\right)^2 - m_X^2} \\
        &= \frac{1}{8\pi}\sqrt{1 - \left(\frac{2m_X}{m_h}\right)^2}.
    \end{align*}
    Putting it all together, the partial width is
    \[ \Gamma_{h\to XX} = \frac{\kappa^2}{16\pi m_h}\sqrt{1 - \left(\frac{2m_X}{m_h}\right)^2} \]

    \item Assume the possible SM decays are enumerated. The $i^\text{th}$ branching ratio $BR_i$ is given by
    \[ BR_i = \frac{\Gamma_i}{\sum_j\Gamma_j}. \]
    Adding a new BSM decay, we get
    \[ BR_i \to \tilde{BR}_i = \frac{\Gamma_i}{\sum_j\Gamma_j + \Gamma_{h\to XX}}. \]
    That is, any particular branching ratio is changed by a constant factor of
    \[ \frac{\tilde{BR}_i}{BR_i} = \frac{\sum_j\Gamma_j}{\sum_j\Gamma_j + \Gamma_{h\to XX}} \]

    \item Let $\Gamma_0 = \sum_j\Gamma_j$. Then,
    \[ \frac{\Gamma_0}{\Gamma_0 + \Gamma_{h\to XX}} = \frac{1}{4} \implies \Gamma_{h\to XX} = 3\Gamma_0 \]
    For a massless $X$, the partial width is simply $\kappa^2/4\pi m_h$, so
    \[ \frac{\kappa^2}{16\pi m_h} = 3\Gamma_0 \implies \kappa = \sqrt{48\pi\Gamma_0m_h} \]
    for a 125 GeV Higgs with a width of 3.2 MeV, this gives a $\kappa$ of
    \[ \kappa \approx 7.8 GeV \]
\end{enumerate}

\section{Problem 2}
\begin{figure}[H]
    \centering
    \resizebox{0.65\textwidth}{!}{
        \feynmandiagram [horizontal=i1 to p1] {
        i1[particle=$h$] -- [scalar] p1 -- [scalar] f1[particle=$W_\mu^+$],
        p1 -- [scalar] f2[particle=$W_\mu^-$],
        };
        \feynmandiagram [horizontal=i1 to p1] {
        i1[particle=$h$] -- [scalar] p1 -- [scalar] f1[particle=$Z_\mu^0$],
        p1 -- [scalar] f2[particle=$Z_\nu^0$],
        };
    }
    \caption{$h \to W^+W^-$ and $h \to ZZ$ decays.}
    \label{fig2}
\end{figure}

\begin{enumerate}[label=(\alph*)]
    \item The Feynman diagrams in figure \ref{fig2} correspond to the SM lagrangian terms
    \[ \dots + \frac{2m_w^2}{v}hW_\mu^+W^{-\mu} + \frac{m_Z^2}{v}hZ_\mu Z^\mu + \dots \]
    given in Peskin (21.11).

    \item The Higgs is spin-0, so to conserve angular momentum the spins of the resulting particles must be anti-aligned. Since, (in the CM frame) the particles momenta are anti-aligned, the helicities must be equal.
    \item The polarization vectors for the $W^+$ are given by
    \[ \epsilon_\pm^\mu = \frac{1}{\sqrt{2}}\mqty(0\\1\\\pm i\\0); \quad \epsilon_0^\mu = \mqty(0\\0\\0\\1). \]
    The specified boost is along the $\hat{z}$ direction. Therefore, only the 0 and 3 components of 4-vectors are affected by it. Clearly then, the $\epsilon_\pm$ polarizations vectors are unaffected. The boost that takes $(m_W, 0, 0, 0)^\mu$ to $(E, 0, 0 k)$ can be expressed in the 0-3 subspace as
    \[ B = \frac{1}{m_W}\mqty(E&k\\k&E). \]
    The boosted longitudinal polarization vector is then 
    \[ \epsilon_0^\mu \to \frac{1}{m_W}\mqty(E&k\\k&E)\mqty(0\\1) = \frac{1}{m_W}\mqty(k\\E), \]
    as desired.

    \item Spatial rotations about the $x$-axis take $\hat{y}\to-\hat{y}$ and $\hat{z}\to-\hat{z}$ The effect on the polarization vectors is then
    \[ \epsilon_\pm^\mu \to \frac{1}{\sqrt{2}}\mqty(0\\1\\\mp i\\0); \quad \epsilon_0^\mu \to \mqty(0\\0\\0\\-1). \]
    Note that, had we rotated around the $y$-axis, the effect for the longitudinal polarization would have been identical, while the transverse polarizations each pick up an overall minus sign.

    \item The matrix element is given by the vertex factor, $2m_W^2/v$, times the product of the polarization vectors. The polarization vector products are given by
    \begin{align*}
        \frac{1}{\sqrt{2}}\mqty(0\\1\\i\\0)_\mu \frac{1}{\sqrt{2}}\mqty(0\\1\\-i\\0)^\mu &= -1 \\
        \frac{1}{m_W}\mqty(k\\0\\0\\E)_\mu\frac{1}{m_W}\mqty(k\\0\\0\\-E) &= \frac{k^2 + E^2}{m_W^2},
    \end{align*}
    giving matrix elements (modulo $i$)
    \[ \mathcal{M}_{++} = \mathcal{M}_{--} = -\frac{2m_W^2}{v}, \]
    \[ \mathcal{M}_{00} = \frac{2}{v}(k^2 + E^2). \]

    \item The total matrix element (modulo $i$) is then
    \[ \mathcal{M} = \mathcal{M}_{++} + \mathcal{M}_{--} + \mathcal{M}_{00} = \frac{2}{v}\left(\magsq{\vec{p}_1} + \frac{1}{4}m_h^2 - m_W^2\right), \]
    with square
    \begin{align*}
        \magsq{\mathcal{M}} &= \frac{4}{v^2}\left[\abs{\vec{p}_1}^4 + 2\magsq{\vec{p}_1}\left(\left(\frac{m_h}{2}\right)^2 - m_W^2\right) + \left(\left(\frac{m_h}{2}\right)^2 - m_W^2\right)^2\right] \\
        &= \frac{4}{v^2}\left[\abs{\vec{p}_1}^4 + \frac{\magsq{\vec{p}_1}m_h^2}{2}\left(1-\left(\frac{2m_W^2}{m_h}\right)^2\right) + \frac{m_h^4}{16}\left(1-\left(\frac{2m_W^2}{m_h}\right)^2\right)^2 \right]
    \end{align*}
    where $\vec{k} \to \vec{p}_1$ for consistence with problem 1, and $E=m_h/2$ has been substituted. The calculate the rate, we will need the phase space integral
    \[ \int\dd\Pi_2 =  \frac{1}{8\pi}\sqrt{1 - \left(\frac{2m_W}{m_h}\right)^2}, \]
    which was calculated in problem 1, as well as
    \[ \int\magsq{\vec{p}_1}\dd\Pi_2, \]
    and 
    \[ \int\abs{\vec{p}_1}^4\dd\Pi_2. \]
    Now,
    \begin{align*}
        \int\magsq{\vec{p}_1}\dd\Pi_2 &= \int \frac{\dd^3\vec{p}_1}{(2\pi)^32E_1}\frac{\dd^3\vec{p}_2}{(2\pi)^32E_2}\magsq{\vec{p}_1}(2\pi)^4\delta^4(p - p_1 - p_2) \\
        &= \frac{1}{4(2\pi)^2}\int\dd^3\vec{p}_1\frac{\magsq{\vec{p}_1}}{E_1^2}\delta(2E_1-m_h) \\
        &= \frac{1}{4\pi}\int\dd\abs{\vec{p}_1}\frac{\abs{\vec{p}_1}^4}{E_1^2}\delta(2E_1-m_h) \\
        &= \frac{1}{4\pi}\frac{\abs{\vec{p}_1}^4}{E_1^2}\frac{E_1}{2\abs{\vec{p}_1}}\Biggr\rvert_{E_1=\frac{m_h}{2}} \\
        &= \frac{1}{8\pi}\frac{\abs{\vec{p}_1}^3}{E_1}\Biggr\rvert_{E_1=\frac{m_h}{2}} \\
        &= \frac{1}{4\pi m_h}\left(\left(\frac{m_h}{2}\right)^2 - m_W^2\right)^{3/2} \\
        &= \frac{m_h^2}{32\pi}\left(1 - \left(\frac{2m_h}{m_W^2}\right)^2\right)^{3/2},
    \end{align*}
    and similarly
    \[ \int\abs{\vec{p}_1}^4\dd\Pi_2 = \frac{m_h^4}{128\pi}\left(1 - \left(\frac{2m_h}{m_W^2}\right)^2\right)^{5/2}. \]
    The decay width is given by
    \[ \Gamma = \frac{1}{2m_h}\int\dd\Pi_2\magsq{\mathcal{M}(h\to W^+W^-)}. \]
    Let $\delta = (1 - (2m_W/m_h)^2)$. Putting it all together, we get
    \begin{align*}
        \Gamma_{h\to W^+W^-} &= \frac{2}{v^2m_h} \left[\frac{m_h^4}{128\pi}\delta^{5/2} + \frac{m_h^2}{2}\delta\frac{m_h^2}{32\pi}\delta^{3/2} + \frac{m_h^4}{16}\delta^2\frac{1}{8\pi}\delta^{1/2}\right] \\
        &= \frac{m_h^3}{64\pi v^2}\delta^{5/2}[1 + 2 + 1] \\
        &= \frac{m_h^3}{16v^2}\left(1 - \left(\frac{2m_h}{m_W}\right)^2\right)^{5/2} \\
        &= \frac{\alpha_Wm_h^3}{4m_W^2}\left(1 - \left(\frac{2m_h}{m_W}\right)^2\right)^{5/2} \\
    \end{align*}
    which has the same $m_h^3$ dependence that Peskin has, but not exactly the right form =(.

    \item For $h \to ZZ$, everything goes through almost identically. The only differences are that $m_W \to m_Z$, and we get an extra factor of $1/2!$ since the final state particles are identical, but that cancels with the extra factor of 2 in the lagrangian. Thus,
    \[ \Gamma_{h \to ZZ} = \frac{\alpha_Wm_h^3}{4m_Z^2}\left(1 - \left(\frac{2m_h}{m_Z}\right)^2\right)^{5/2} \\ \]
    \item The longitudinal polarization gives momentum dependence to the matrix element, so it seems that it should be the longitudinal helicity amplitude that is responsible for the $m_h^3$ growth.
\end{enumerate}

\[ \mathscr{L} = \dots -\lambda\left(\frac{(v + h)^2}{2} + \frac{1}{2}\left(\left(G^0\right)^2 + 2G^+G^-\right) - \frac{v^2}{2}\right)^2 \]
\[ \mathcal{M}(h\to G^0G^0) = ddd \]

\end{document}