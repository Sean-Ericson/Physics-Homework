\documentclass[12pt]{article}

\usepackage[margin=1in]{geometry}
\usepackage{amsmath,amsthm,amssymb}
\usepackage{mathrsfs}
\usepackage{enumitem}
\usepackage{physics}

\newcommand{\magsq}[1]{\big|#1\big|^2}
\newcommand{\avg}[1]{\left<#1\right>}
\newcommand{\fullint}{\int_{-\infty}^\infty}
\newcommand{\fullintd}[1]{\fullint\dd#1\:}
\newcommand{\cint}[2]{\int_{#1}^{#2}}
\newcommand{\cintd}[3]{\cint{#1}{#2}\dd#3\:}
\newcommand{\e}{\mathbf{e}}

\begin{document}
	
\title{Homework 8}
\author{Sean Ericson \\ Phys 632}
\maketitle

\section*{Problem 1}
First, let's identify
\[ \e = e^{\beta\hbar\omega}, \quad \coth(\beta\hbar\omega) = \frac{\e^1 + \e^{-1}}{\e^1 - \e^{-1}}, \quad \csch(\beta\hbar\omega) = \frac{2}{\e^1 - \e^{-1}} \]
as well as
\[ Z = \sum_n\e^{-(n+1/2)} = \e^{-\frac{1}{2}}\sum_n\e^{-n} = \frac{\e^{-\frac{1}{2}}}{1 - \e^{-1}} = \frac{1}{\e^\frac{1}{2} - \e^{-\frac{1}{2}}} \]
Now, neglecting units for the time being,
\begin{align*}
    \rho(x,x') &= \mel{x}{\rho}{x'} \\
    &= Z^{-1}\sum_n \e^{-(n+1/2)}\braket{x}{n}\braket{n}{x'} \\
    &= \frac{1}{\sqrt{\pi}}\left(\e^\frac{1}{2} - \e^{-\frac{1}{2}}\right)\e^{-\frac{1}{2}} e^{-(x^2 + (x')^2)/2}\sum_n\frac{\e^{-n}}{2^n n!}H_n(x)H_n(x') \\
\end{align*}
Letting $z = \e^{-1}$ and applying Mehler's formula gives
\begin{align*}
    \rho(x,x') &= \frac{1}{\sqrt{\pi}}\frac{1 - \e^{-1}}{\sqrt{1 - \e^{-2}}}e^{-(x^2 + (x')^2)/2}\exp\left[\frac{2xx'\e^{-1} - (x^2 + (x')^2)\e^{-2}}{1 - \e^{-2}}\right] \\
    &= \frac{1}{\sqrt{\pi}}\frac{\e^{\frac{1}{2}}(\e^{\frac{1}{2}} - \e^{-\frac{1}{2}})}{\e^{\frac{1}{2}}\sqrt{\e^1 - \e^{-1}}}\exp\left[\frac{2xx'}{\e^{1} - \e^{-1}} - (x^2 + (x')^2)\left(\frac{\e^{-1}}{\e^1 - \e^{-1}} + \frac{1}{2}\right)\right] \\
    &= \frac{1}{\sqrt{\pi}}\frac{\e^{\frac{1}{2}} - \e^{-\frac{1}{2}}}{\sqrt{(\e^{\frac{1}{2}}-\e^{-\frac{1}{2}})(\e^{\frac{1}{2}}+\e^{-\frac{1}{2}})}}\exp[xx'\csch(\beta\hbar\omega) - \frac{1}{2}(x^2 + (x')^2)\coth(\beta\hbar\omega)] \\
    &= \frac{1}{\sqrt{\pi\coth(\beta\hbar\omega/2)}}\exp[xx'\csch(\beta\hbar\omega) - \frac{1}{2}(x^2 + (x')^2)\coth(\beta\hbar\omega)] \\
\end{align*}
Restoring units,
\[ \rho(x,x') = \frac{1}{\sqrt{\pi x_s^2 \coth(\beta\hbar\omega/2)}}\exp[\frac{xx'}{x_s^2}\csch(\beta\hbar\omega) - \frac{(x^2 + (x')^2))}{2x_s^2}\coth(\beta\hbar\omega)] \]
For the position uncertainty, $\rho(x,x)$ is a gaussian, so we can immediately read off the  uncertainty as
\[ \sigma_x = \sqrt{\frac{1}{2}x_s^2\coth(\beta\hbar\omega/2)} = \sqrt{\frac{\hbar\coth(\beta\hbar\omega/2)}{2m\omega}} \]
As $T\to 0$, this becomes the typical harmonic oscillator ground state position uncertainty,
\[ \sigma_x = \sqrt{\frac{\hbar}{2m\omega}}. \]
As $T\to\inf$, $\coth(\beta\hbar\omega/2)\to 2/\beta\hbar\omega$, and the uncertainty approaches
\[ \sigma_x = \sqrt{\frac{k_BT}{m\omega^2}} \]
as expexted by the equipartition theorem, $\frac{1}{2}m\omega^2\sigma_x^2 = \frac{1}{2}k_bT$.


\section*{Problem 2}
Before Alice performs her measurement, the three qubits are in the state
\[ \ket{\psi_{\text{TAB}}} = \frac{1}{2}\left[\ket{1 1}\ket{\psi_{11}} + \ket{1 0}\ket{\psi_{10}} + \ket{01}\ket{\psi_{01}} - \ket{00}\ket{\psi_{00}}\right] \]
where
\begin{align*}
    \ket{\psi_{11}} &= c_0\ket{1} + c_1\ket{0} = \sigma_x\ket{\psi_{10}} \\
    \ket{\psi_{10}} &= c_1\ket{1} + c_0\ket{0} = \mathcal{I}\ket{\psi_{10}} \\
    \ket{\psi_{01}} &= c_0\ket{1} - c_1\ket{0} = i\sigma_y\ket{\psi_{10}} \\
    \ket{\psi_{00}} &= c_1\ket{1} - c_0\ket{0} = \sigma_z\ket{\psi_{10}}
\end{align*}
After Bob hears about Alice's measurement result and performs the necessary unitary operation, his state is
\begin{align*}
    \rho_B &= (1-\epsilon)\dyad{\psi_{10}} + \frac{\epsilon}{3}\left(\dyad{\psi_{11}} + \dyad{\psi_{01}} + \dyad{\psi_{00}} \right) \\
    &= (1-\epsilon - \frac{\epsilon}{3})\dyad{\psi_{10}} + \frac{\epsilon}{3}\left(\dyad{\psi_{11}} + \dyad{\psi_{01}} + \dyad{\psi_{00}} + \dyad{\psi_{10}} \right) \\
    &= (1 - \frac{4}{3}\epsilon)\dyad{\psi_{10}} + \frac{4}{3}\epsilon\mathcal{I}
\end{align*}
With a matrix representation the $\{\ket{1}, \ket{0}\}$ basis of
\[ \mqty((1 - \frac{4}{3}\epsilon)\magsq{c_1} + \frac{4}{3}\epsilon & (1-\frac{4}{3}\epsilon)c_0^*c_1 \\ (1-\frac{4}{3}\epsilon)c_0c_1^* & (1 - \frac{4}{3})\epsilon\magsq{c_0} + \frac{4}{3}\epsilon ) \]

\section*{Problem 3}
The mean-squared separation is given by
\begin{align*}
    \ev{(x_A - x_B)^2} &= \ev{x^2}_A + \ev{x^2}_B - 2\ev{x}_A\ev{x}_B \mp 2\magsq{\mel{\psi_1}{x}{\psi_2}} \\
    &= \sigma_x^2(n) + \sigma_x^2(n') \mp 2\magsq{\mel{n}{x}{n'}}
\end{align*}
where the last term is + for fermions, - for bosons, and 0 for distinguishable particles. The position-variances are simply
\[ \sigma_x^2 = \frac{\hbar(n + 1/2)}{m\omega_0} \]
and the overlap is
\begin{align*}
    \mel{n}{x}{n'} &= \frac{1}{\sqrt{2}}\mel{n}{a + a^{\dag}}{n'} \\
    &= \frac{1}{\sqrt{2}}\left(\sqrt{n+1}\delta_{n,n'-1} + \sqrt{n}\delta_{n, n'+1}\right)
\end{align*}
So, for distinguishable particles we have
\[ \ev{(x_A - x_B)^2} = \frac{\hbar(n + n' + 1)}{m\omega_0} \]
For indistinguishable bosons,
\[ \ev{(x_A - x_B)^2} = \frac{\hbar(n + n' + 1)}{m\omega_0} - \frac{1}{\sqrt{2}}\left(\sqrt{n+1}\delta_{n,n'-1} + \sqrt{n}\delta_{n, n'+1}\right)  \]
For indistinguishable fermions,
\[ \ev{(x_A - x_B)^2} = \frac{\hbar(n + n' + 1)}{m\omega_0} + \frac{1}{\sqrt{2}}\left(\sqrt{n+1}\delta_{n,n'-1} + \sqrt{n}\delta_{n, n'+1}\right)  \]


\section*{Problem 4}
\begin{enumerate}[label=(\alph*)]
    \item Parahelium should have the lower energy. Since the electrons' spins are antisymmetric, their spacial degree of freedom must be bosonic, so the can both occupy the lowest energy state.
    \item For the delium atom, the electrons' wavefunctions need not be symmetric or antisymmetric, so their degree of localization should be betwen that of the para/orthohelium cases, and its energy should also be between that of para/orthohelium.
\end{enumerate}


\section*{Problem 5}
\[ \Tr[Q] = p_0 - p_1; \qquad \det[Q] = \left[(p_0-p_1)^2 - 1\right]\sin^2\theta\cos^2\theta \]
\begin{align*}
    q_{\pm} &= \frac{\Tr[Q]}{2} \pm \sqrt{\left(\frac{\Tr[Q]}{2}\right)^2 - \det[Q]} \\
    &= \frac{p_0-p_0}{2} \pm \sqrt{\left(\frac{p_0-p_1}{2}\right)^2 - \left[(p_0-p_1)^2 - 1\right]\sin^2\theta\cos^2\theta} \\
    &= \frac{1}{2}\left(p_0 - p_1 \pm \sqrt{(p_0-p_1)^2 - \left[(p_0-p_1)^2 - 1\right]\sin^2(2\theta)}\right) \\
    &= \frac{1}{2}\left(p_0 - p_1 \pm \sqrt{(p_0-p_1)^2 - \left[(p_0-p_1)^2 - 1\right]\left(1-\cos^2(2\theta)\right)}\right) \\
    &= \frac{1}{2}\left(p_0 - p_1 \pm \sqrt{1 - \cos^2(2\theta) - (p_0 - p_1)^2\cos^2(2\theta)}\right) \\
    &= \frac{1}{2}\left(p_0 - p_1 \pm \sqrt{1 + (1 - (p_0 - p_1)^2)\cos^2(2\theta)} \right) \\
    &= \frac{1}{2}\left(p_0 - p_1 \pm \sqrt{1 - 4p_0p_1\cos^2{2\theta}}\right)
\end{align*}


\end{document}