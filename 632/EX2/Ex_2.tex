\documentclass[12pt]{article}

\usepackage[margin=1in]{geometry}
\usepackage{amsmath,amsthm,amssymb}
\usepackage{enumitem}
\usepackage{physics}

\newcommand{\magsq}[1]{\big|#1\big|^2}
\newcommand{\avg}[1]{\left<#1\right>}
\newcommand{\fullint}{\int_{-\infty}^\infty}
\newcommand{\fullintd}[1]{\fullint\dd#1\:}
\newcommand{\cint}[2]{\int_{#1}^{#2}}
\newcommand{\cintd}[3]{\cint{#1}{#2}\dd#3\:}

\begin{document}
	
\title{Exercise Set 2}
\author{Sean Ericson \\ Phys 632}
\maketitle

\section*{Exercise 1}
First let's calculate the derivative
\[ \partial_x\psi = \left(\pm\frac{i\eta}{\hbar} \sqrt{p(x)} -\frac{\eta p'(x)}{2(p(x))^{3/2}}\right) \exp[\pm \frac{i}{\hbar}\int^x p(x') \dd x' ] \]
Multiplying by $\psi^*$ will cancel the exponential, so
\begin{align*}
    j(x) & = \frac{\hbar}{m}\Im\left[\frac{\eta}{\sqrt{p(x)}}\left(\pm\frac{i\eta}{\hbar} \sqrt{p(x)} -\frac{\eta p'(x)}{2(p(x))^{3/2}}\right)\right] \\
    & = \pm \frac{\eta^2}{m}
\end{align*}


\section*{Exercise 2}
The turning points are given by
\[ V(x_{1,2}) = E \implies m\omega^2x_{1,2}^2 = E \implies x_{1,2} = \mp\sqrt{\frac{E}{m\omega^2}} \]
Now,
\begin{align*}
    \oint p(x)\dd x & = \cint{-\sqrt{\frac{E}{m\omega^2}}}{\sqrt{\frac{E}{m\omega^2}}} \sqrt{2mE - 2m^2\omega^2x^2} \dd x \\
    & = 4m\omega \cint{-\sqrt{\frac{E}{m\omega^2}}}{\sqrt{\frac{E}{m\omega^2}}} \sqrt{\frac{E}{m\omega^2} - x^2} \\
    & = 4m\omega \frac{E}{m\omega^2}\frac{\pi}{2} \\
    & = \frac{2\pi E}{\omega}  = \left(n + \frac{1}{2}\right)h
\end{align*}
\[ \implies E = \left(n + \frac{1}{2}\right)\hbar\omega \]    


\end{document}