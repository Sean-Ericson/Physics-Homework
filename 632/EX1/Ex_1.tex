\documentclass[12pt]{article}

\usepackage[margin=1in]{geometry}
\usepackage{amsmath,amsthm,amssymb}
\usepackage{enumitem}
\usepackage{physics}

\newcommand{\magsq}[1]{\big|#1\big|^2}
\newcommand{\avg}[1]{\left<#1\right>}
\newcommand{\fullint}{\int_{-\infty}^\infty}
\newcommand{\fullintd}[1]{\fullint\dd#1\:}
\newcommand{\cint}[2]{\int_{#1}^{#2}}
\newcommand{\cintd}[3]{\cint{#1}{#2}\dd#3\:}

\begin{document}
	
\title{Exercise Set 1}
\author{Sean Ericson \\ Phys 632}
\maketitle

\section*{Exercise 1}
\begin{enumerate}[label=(\alph*)]
    \item We want to approximate
    \[ n! = \cintd{0}{\infty}{t} e^{-t + n\log{t}}. \]
    First we expand the function in the exponential about it's maximum. The maximum occurs at
    \[ \dv{t}(-t + n\log{t}) = 0 \implies \frac{n}{t} - 1 = 0 \implies t = n. \]
    Expanding about that point gives
    \[ -(t - n) + n\log(t - n) \approx -n - n\log{n} - \frac{(t-n)^2}{2n} \]

    \item Plugging this back into the exponential we see that
    \begin{align*}
        n! & \approx e^{-n - n\log{n}}\cintd{0}{\infty}{t}e^{-\frac{(t-n)^2}{2n}} \\
        & \approx \left(\frac{n}{e}\right)^n \sqrt{2\pi n}
    \end{align*}
\end{enumerate}



\section*{Exercise 2}
\begin{enumerate}[label=(\alph*)]
    \item Given that $x = \frac{1}{2}\frac{F}{m}t^2$, we can see that $F \propto \frac{mx}{t^2}$ thus
    \[ \Delta F \propto \Delta x \frac{m}{\tau^2} = \sqrt{\frac{\hbar\tau}{m}}\frac{m}{\tau^2} = \sqrt{\frac{\hbar m}{\tau^3}} \]

    \item Multiplying the expression for $\Delta F$ by a quantity with dimensions of time ($\tau$) gives
    \[ \Delta p \propto \tau\sqrt{\frac{\hbar m}{\tau^3}} = \sqrt{\frac{\hbar m}{\tau}} \]
\end{enumerate}
    


\end{document}