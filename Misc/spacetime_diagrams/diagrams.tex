\documentclass[12pt]{article}

\usepackage[margin=1in]{geometry}
\usepackage{amsmath,amsthm,amssymb}
\usepackage{nccmath}
\usepackage{mathtools}
\usepackage{mathrsfs}
\usepackage{enumitem}
\usepackage{physics}
\usepackage{tensor}
\usepackage{ifthen}

\usepackage{tikz}
\usetikzlibrary{calc,decorations.markings,patterns,math}


\begin{document}

\title{Spacetime}
\author{Sean Ericson}
\maketitle

\section*{}
\begin{figure}
    \centering
    \begin{tikzpicture}
        \tikzmath{\rad = 5; \vel = 0.6; \gam = 1;}

        % Unprimed Axes
        \draw[<->, thick] (0,-\rad)--(0,\rad) node[above]{$t$};
        \draw[<->, thick] (-\rad,0)--(\rad,0) node[right]{$x$};

        % Primed Axes
        \draw[->] (0,0)--({\gam*\vel*\rad}, {\gam*\rad}) node[above]{$t'$};
        \draw[->] (0,0)--({\gam*\rad}, {\gam*\vel*\rad}) node[right]{$x'$};
        
        \foreach \i in {-\rad,...,\rad}{
            % const x'
            %\draw[very thin, gray] ({\gam*\i - \gam*\rad*\vel}, {-\gam*\rad + \gam*\i*\vel})--({\gam*\i + \gam*\rad*\vel}, {\gam*\rad + \gam*\i*\vel});

            % const t'
            %\draw[very thin, gray] ({-\gam*\rad + \gam*\vel*\i}, {\gam*\i - \gam*\vel*\rad})--({\gam*\rad + \gam*\vel*\i}, {\gam*\i + \gam*\vel*\rad});
        }

        % Null line
        %\draw[dashed] ({-\gam*\rad - \gam*\rad*\vel}, {-\gam*\rad - \gam*\rad*\vel})--({\gam*\rad + \gam*\rad*\vel}, {\gam*\rad + \gam*\rad*\vel});
        %\draw[dashed] ({\gam*\rad - \gam*\rad*\vel}, {-\gam*\rad + \gam*\rad*\vel})--({-\gam*\rad + \gam*\rad*\vel}, {\gam*\rad - \gam*\rad*\vel});

        % Event A (origin)
        \coordinate (A) at (0,0);
        \filldraw (A) circle[radius=1.5pt];
        \node[left] at (A) {A};

        % Event B
        \coordinate (B) at (3,1.75);
        \filldraw (B) circle[radius=1.5pt];
        \node[right] at (B) {B};

        % Event C
        \coordinate (C) at (4,2);
        \filldraw (C) circle[radius=1.5pt];
        \node[right] at (C) {C};

        % parallelogram
        \draw[dashed, blue]

    \end{tikzpicture}
    \caption{}
    \label{fig1}
\end{figure}





\end{document}