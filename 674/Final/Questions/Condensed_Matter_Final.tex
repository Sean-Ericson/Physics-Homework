\documentclass[12pt]{article}

\usepackage[margin=1in]{geometry}
\usepackage{amsmath,amsthm,amssymb}
\usepackage{mathtools}
\usepackage{mathrsfs}
\usepackage{enumitem}
\usepackage{physics}
\usepackage{tensor}
\usepackage{array}

\usepackage{tikz}
\usetikzlibrary{calc,decorations.markings,patterns}

\newcommand{\magsq}[1]{\big|#1\big|^2}
\newcommand{\avg}[1]{\left<#1\right>}
\newcommand{\fullint}{\int_{-\infty}^\infty}
\newcommand{\fullintd}[1]{\fullint\dd#1\:}
\newcommand{\cint}[2]{\int_{#1}^{#2}}
\newcommand{\cintd}[3]{\cint{#1}{#2}\dd#3\:}

\begin{document}

\title{PHYS 674 Condensed Matter \\ Final Exam}
\date{Due: Thursday, December 7, 12:00pm}
\maketitle

\section*{Problem 1}
Consider an $O(n)$ model with a ``cubic symmetry breaking term'' $g$, by which I mean the following model:
\begin{equation*}
    H = \int \dd^dr \left[\frac{t}{2}\abs{\vec{M}}^2 + u\abs{\vec{M}}^4 + g\sum_{\alpha=1}^n M_\alpha^4 + \frac{c}{2}\abs{\nabla M}^2\right]
    \tag{1.1}
\end{equation*}
\begin{enumerate}[label=(\alph*)]
    \item Derive the RG recursion relations for this model to one loop order.
    \item Find the fixed points
    \item Identify the fixed point that controls the transition.
    \item Calculate the critical exponents $\alpha$, $\beta$, and $\nu$ to $O(\epsilon = 4-d)$
\end{enumerate}
Hint: Your answers to parts (b), (c), and (d) could be qualitatively different for different values of $n$.

\section*{Problem 3}
Suppose we apply a magnetic field to an X-Y model, this amounts to adding a symmetry breaking term
\begin{equation*}
    \Delta H = -\vec{h}\cdot\sum_i \vec{S}_i
    \tag{3.1}
\end{equation*}
to the standard X-Y Hamiltonian.
\begin{enumerate}[label=(\alph*)]
    \item Write the continuum version of this model, ignoring irrelevant terms.
    \item Show that the model includes a term
    \begin{equation*}
        \Delta H = - \int \dd^dr\cos(\theta(\vec{r}))
        \tag{3.2}
    \end{equation*}
    \item Expand the cosine to all orders in $\theta$, and represent each term by a (schematic) Feynman graph.
    \item Derive RG recursion relations for $a_n$, the coefficient of $\theta^n$ in the above expansion to \underline{linear} order in the $a_n$. Show that, for the correct choice of $\chi_\theta$, the rescaling exponent for $\theta$, the entire series can be resummed to give a $-h(l)\cos(\theta'(\vec{r'}))$, with a renormalized $h(l)$; and thereby derive the recursion relation for $h(l)$ to this order.
    \item Derive the recursion relation for $K(l)$ to this order, using the above choice of $\chi_\theta$.
    \item Show that, for $d>2$, $h(l\to\infty)\to\infty$ (at least, as far as this perturbative calculation can tell). What does this imply about the $\abs{\vec{q}} \ll \Lambda$ limit of the fluctuations $\avg{\abs{\theta(q)}^2}$?
    \item Show that, in $d=2$, there is a phase transition at some critical temperature $T_c$ in this model, in the sense that qualitatively different scaling behaviors apply for $T<T_c$ and $T>T_c$. Calculate $T_c$ in terms of $K$.
    \item Calculate $\avg{\cos\theta(r)}$ in a $d=2$ $L\times L$ square system, taking $h=0$.
\end{enumerate}


\end{document}