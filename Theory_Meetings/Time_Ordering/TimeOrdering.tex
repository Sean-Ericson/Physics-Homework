\documentclass{beamer}

\usetheme{Malmoe}
\usecolortheme{spruce}

\usepackage{amsmath,amsthm,amssymb}
\usepackage{mathtools}
\usepackage{mathrsfs}
\usepackage{enumitem}
\usepackage{physics}

%Information to be included in the title page:
\title{$\mathscr{T}$: The Chronological ``Operator''}
\author{Sean Ericson}
\institute{UO}
\date{Group meeting, June 2, 2023}
\titlegraphic{\includegraphics[scale=0.75]{seal.jpg}}

\begin{document}

\frame{\titlepage}

\begin{frame}
\frametitle{What is $\mathscr{T}$? A few different names...}
\begin{itemize}
    \item<1-> Time Ordering Symbol
    \begin{itemize}
        \item<1-> ``...the symbol means the product of operators to its right is to be ordered, not as written, but with operators at later times to the left of those at earlier times.'' -Srednicki, QFT
    \end{itemize}
    \item<2-> Time Ordered Product
    \begin{itemize}
        \item<2-> ``...the product with factors arranged so that the one with the latest time argument is placed leftmost, the next-latest is placed next to leftmost, and so on.'' -Wineberg, The Quantum Theory of Fields
    \end{itemize}
    \item<3-> The Chronological Operator
    \begin{itemize}
        \item ``...reorders its argument such that the times are in increasing order from right to left.'' -Steck, QM
    \end{itemize}
\end{itemize}
\end{frame}

\begin{frame}
\frametitle{The Question: What is it?}
    \alert{\textbf{Is it...}}
    \begin{itemize}
        \item[\textbullet]<1-> an operator?
        \item[\textbullet]<2-> a super-operator?
        \item[\textbullet]<3-> a meta-operator?
        \item[\textbullet]<4-> just a function?
        \item[\textbullet]<5-> just a rewrite rule? 
    \end{itemize}
\end{frame}

\begin{frame}
\frametitle{Defining $\mathscr{T}$}
\alert{\textbf{A Mathematical Definition}:}
\begin{itemize}
    \item[\textbullet]<2-> Let $\mathcal{O}_{\mathcal{H}}$ denote the algebra of linear operators on a Hilbert space $\mathcal{H}$.
    \item[\textbullet]<3-> Let $\mathcal{D}_n$ deonte the set of time-dependent operators that depend on $n$ time coordinates $\{\sigma|\;\sigma:\mathcal{I}^n\to\mathcal{O}_{\mathcal{H}}\}$ where $\mathcal{I}$ is some interval of $\mathbb{R}$.
    \item[\textbullet]<4-> Consider $A, B \in \mathcal{D}_1$
    \item[\textbullet]<5-> Define time-dependent operators $AB, BA, T\{AB\} \in \mathcal{D}_2$ via
    \begin{itemize}
        \item<6-> $(AB)(t_1, t_2) = A(t_1)B(t_2)$
        \vspace{0.1cm}
        \item<7-> $(BA)(t_1, t_2) = B(t_1)A(t_2)$
        \vspace{0.1cm}
        \item<8-> $T\{AB\}(t_1, t_2) = \begin{cases}
			(AB)(t_1, t_2), & t_1 > t_2 \\
            (BA)(t_2, t_1), & t_1 < t_2
		 \end{cases}$
    \end{itemize} 
\end{itemize}
\end{frame}

\begin{frame}
    \frametitle{Defining $\mathscr{T}$}
    \alert{\textbf{A (More General) Mathematical Definition}:}
    \begin{itemize}
        \item[\textbullet]<2-> Consider time-dependent operators $\sigma_i(t_i)\in\mathcal{D}_1$ $(i=1\ldots n)$.
        \item[\textbullet]<3-> Define the (bosonic or fermionic) $\mathscr{T}:\mathcal{D}_1^n\to \mathcal{D}_n$ via
        \begin{align*}
            \mathscr{T}\{\prod_{i=1}^{n} \sigma_i(t_i)\} &= f(\sigma_1, \sigma_2, \ldots, \sigma_n) \\
            &= \sum_{\pi\in S_n}\epsilon(\pi)\left(\prod_{j=1}^{n-1}\theta(t_{\pi_j} - t_{\pi_{j+1}})\right)\prod_{i=1}^n\sigma_{\pi_i}(t_{\pi_i})
        \end{align*}
        where 
        \begin{itemize}
            \item[\textbullet]<4-> $S_n$ is the symmetric group on $n$ elements,
            \item[\textbullet]<5-> $\epsilon(\pi) = \begin{cases}
                1 & \text{(bosonic)} \\
                \text{sgn}(\pi) & \text{(fermionic)}
             \end{cases}$
            \item[\textbullet]<6-> and $\theta(t)$ is the Heaviside step function.
        \end{itemize}
    \end{itemize}
\end{frame}

\begin{frame}
\frametitle{Answering the Question}
\alert{\textbf{So, is it an operator?}}
\begin{itemize}
    \item[\textbullet]<2-> Well, what exactly is an operator??
    \item[\textbullet]<3-> From the wikipedia page ``Operator (Mathematics)'': 
    \begin{itemize}
        \item[\textbullet]<3-> ``...there is no general definition of an operator, but the term is often used in place of `function' ''.
    \end{itemize}
    \item[\textbullet]<4-> In the context of quantum mechanincs, we mean functions $\sigma: \mathcal{H} \to \mathcal{H}'$ that map elements from one Hilbert space to another (perhaps the same).
    \item[\textbullet]<5-> For our purposes, we can more generally just consider operators to be functions between vector spaces.
\end{itemize}

\end{frame}

\begin{frame}
    \frametitle{Answering the Question}
    \alert{\textbf{So, is it an operator?!}}
    \begin{itemize}
        \item[\textbullet]<2-> By the definition of ``operator'' given above, \textbf{yes}!!
        \begin{itemize}
            \item[\textbullet]<3-> $\mathcal{D}_1^n$ and $\mathcal{D}_n$ are both vector spaces, and
            \item[\textbullet]<4-> $\mathscr{T}$ maps elemets from $\mathcal{D}_1^n$ to $\mathcal{D}_n$, so 
            \item[\textbullet]<5-> it's an operator!
        \end{itemize}
    \end{itemize}
\end{frame}

\begin{frame}
\frametitle{Why Ask in the First Place?}
\alert{\textbf{Why would it not be an operator?}}
\begin{itemize}
    \item[\textbullet]<2-> $\mathscr{T}$ is not an element of $\mathcal{O}_{\mathcal{H}}$, the set of linear operators on the Hilbert space.
    \begin{itemize}
        \item[\textbullet]<3-> Likely what most people implicitly mean when they say ``operator''.
    \end{itemize}
    \item[\textbullet]<4-> It's also not a \textit{superoperator}: $\Sigma: \mathcal{O}_{\mathcal{H}} \to \mathcal{O}_{\mathcal{H}}$ 
    \begin{itemize}
        \item[\textbullet]<5-> Even though it \textit{looks} like one!
    \end{itemize}
\end{itemize}
\end{frame}

\begin{frame}
\frametitle{Further Classifications}
\alert{\textbf{What about those other terms?}}
\begin{itemize}
    \item<2-> Meta-Operator?
    \begin{itemize}
        \item[\textbullet]<3-> From wikipedia: "...a specific operation over a combination of operators, as in the example of path-ordering. A meta-operator is generally neither an operator (a linear transform on the vector space) nor a superoperator (a linear transform on the space of operators)."
        \item[\textbullet]<4-> The Chronological Operator is just a special case of the ``path-ordering'' operator, and so is a meta-operator. 
    \end{itemize}
\end{itemize}
\end{frame}

\begin{frame}
    \frametitle{Further Classifications}
    \alert{\textbf{What about those other terms?}}
    \begin{itemize}
        \item<2-> ``Just a function?''
        \begin{itemize}
            \item[\textbullet]<3-> Whether or not it's a function depends on the spaces on which it acts.
            \item[\textbullet]<4-> If we try to think of it as a superoperator, we can see that it fails to even be a function:
            \item[\textbullet]<5-> Consider $\sigma_1(t_1), \sigma_2(t_2), \sigma_3(t_3), \sigma_4(t_4) \in \mathcal{D}_1$ with $t_1<t_2<t_3<t_4$
            \item[\textbullet]<6-> Let $\sigma_2(t_2)\sigma_1(t_1) = \sigma_3(t_3)\sigma_4(t_4) = A$
            \item[\textbullet]<7-> Also let $\sigma_4(t_4)\sigma_3(t_3) = B$. Then,
            \item[]<8-> \[ \mathscr{T}\{\sigma_2(t_2)\sigma_1(t_1)\} = \sigma_2(t_2)\sigma_1(t_1) = A \] 
            \item[]<9-> but \begin{align*}
                \mathscr{T}\{\sigma_3(t_3)\sigma_4(t_4)\} &= \sigma_4(t_4)\sigma_3(t_3) = B \\
                &= \mathscr{T}\{A\} = A
            \end{align*}  
        \end{itemize}
    \end{itemize}
\end{frame}

\begin{frame}
    \frametitle{Further Classifications}
    \alert{\textbf{What about those other terms?}}
    \begin{itemize}
        \item<2-> ``Just a rewrite rule?''
        \begin{itemize}
            \item[\textbullet]<3->
        \end{itemize}
    \end{itemize}
\end{frame}

\end{document}