\documentclass{beamer}

\usetheme{Malmoe}
\usecolortheme{spruce}

\usepackage{amsmath,amsthm,amssymb}
\usepackage{mathtools}
\usepackage{mathrsfs}
\usepackage{enumitem}
\usepackage{physics}

\newcommand{\avg}[1]{\left<#1\right>}

%Information to be included in the title page:
\title{Mathematics of Physical Quantities and Units:}
\subtitle{Dimensional Analysis and The $\Pi$ Theorem}
\author{Sean Ericson}
\institute{UO}
\date{Theory meeting, October 27, 2023}
\titlegraphic{\includegraphics[scale=0.75]{seal.jpg}}

\begin{document}

\frame{\titlepage}

\begin{frame}
\frametitle{Scalars, Physical Quantities, and Dimensions}
\begin{itemize}
    \item<1-> Physical quantities ($\mathcal{P}$), their units ($\mathcal{D}$), and the set of scalars ($\mathcal{S}$) all form $\mathbb{Q}$-vector spaces.
    \begin{itemize}
        \item<2-> Note that we consider $\mathcal{S}$ as a 1-dimensional subspace of $\mathcal{P}$.
    \end{itemize}
    \item<3-> We can describe their relationship via the \textit{short exact sequence}
    \item<4-> \[ \mathcal{E} = \left( \mathcal{S} \xrightarrow{i} \mathcal{P} \xrightarrow{d} \mathcal{D} \right), \] where
    \begin{itemize}
        \item<5-> $i$ is the inclusion map of $\mathcal{S}$ into $\mathcal{P}$,
        \item<6-> $d$ is a surjection, and
        \item<7-> $\ker[d] = \text{im}[i]$.
    \end{itemize}
\end{itemize}
\end{frame}

\begin{frame}
\frametitle{A Quick Demonstration of the Notation}
    \begin{itemize}
        \item<1-> Denote preimages of $d$ via $\avg{D} \coloneqq d^{-1}(D)$ for $D \in \mathcal{D}$.
        \begin{itemize}
            \item<2-> i.e. ``the set of quantities with units D''
            \item<3-> Note that $\avg{0} = \mathcal{S}$.
        \end{itemize}
        \item<4-> For all $q_1, q_2 \in \avg{D}$, we have that $q_1 - q_2 \in \mathcal{S}$
        \begin{itemize}
            \item<5-> $d(q_1 - q_2) = d(q_1) - d(q_2) = D - D = 0$
        \end{itemize}
        \item<6-> For any $\lambda \in \mathcal{S}$ and $q \in \avg{D}$, we have that $\lambda + q \in \avg{D}$
        \begin{itemize}
            \item<7-> $d(\lambda + q) = d(\lambda) + d(q) = D + 0 = D$
        \end{itemize}
    \end{itemize}
\end{frame}

\begin{frame}
\frametitle{Splittings (i.e. Choosing Base Units)}
\begin{itemize}
    \item<1-> A linear map $U:\mathcal{D}\rightarrow\mathcal{P}$ such that $d(U(D)) = D \quad \forall D \in \mathcal{D}$ is called a \textit{linear splitting}.
    \item<2-> Any such splitting induces a (linear) map $V:\mathcal{P}\rightarrow\mathcal{S}$ via  $V(q) = q - U(d(q))$
    \item<3-> We can define the map $\chi : \mathcal{S} \times \mathcal{D} \rightarrow \mathcal{P}$  by \[ \chi(\lambda, D) = \lambda + U(D), \]
    \item<4-> and its inverse \[ \chi^{-1}(q) = (V(q), d(q)) \]
    \item<5-> $\chi$ as given is a bijection, hence $\mathcal{S}\times\mathcal{D} \simeq \mathcal{P}$.
\end{itemize}
\end{frame}

\begin{frame}
    \frametitle{$G(\mathcal{E})$-invariance}
    \begin{itemize}
        \item<1-> We denote by $G(\mathcal{E})$ the group of linear maps from $\mathcal{D}$ to $\mathcal{S}$, $\text{Lin}[\mathcal{D}, \mathcal{S}]$.
        \item<2-> An n-ary relation $F \subseteq \mathcal{P}^n$ is said to be \textit{$G(\mathcal{E})$-invariant} if
        \item<3-> \[ (q_1, \cdots, q_n) \in F \implies (q_1 + \theta(d(q_1)), \cdots, q_n + \theta(d(q_n))) \in F \]
        \item<4-> \textit{Example: } $lwh - V = 0$
    \end{itemize}
\end{frame}

\begin{frame}
    \frametitle{The $\Pi$ Theorem for Relations}
    \alert{\textit{Statement}}
    \begin{itemize}
        \item<2->Consider a $G(\mathcal{E})$-invariant $F \subseteq \avg{E_1}, \cdots, \avg{E_m}, \avg{D_1}, \cdots, \avg{D_n}$ where the $\{E_i\}$ are linearly independent and $D_j = \sum k_{ji}E_i$.
        \item<3-> Then, there exists $\Phi \subseteq \mathcal{S}^n$ such that \[ (p_1, \cdots, p_m, q_1, \cdots, q_n) \in F \] \[ \implies (q_1 - \sum k_{1i}p_i, \cdots, q_n - \sum k_{ni}p_i) \in \Phi \] 
    \end{itemize} 
\end{frame}

\begin{frame}
    \frametitle{The $\Pi$ Theorem for Relations}
    \alert{\textit{Proof}}
    \begin{itemize}
        \item<2-> Let $U_0$ be an arbitrary splitting of $d$.
        \item<3-> Define $\Phi = \{ (\lambda_1, \cdots, \lambda_n) \in \mathcal{S}^n | (U_0(E_1), \cdots, U_0(E_m), \lambda_1 + U_0(D_1), \cdots, U_0(D_n)) \in F \}$
        \item<4-> The set $\{E_i\}$ can be extended to a basis for $\mathcal{D}$. We can then define a new splitting $U$ such that $U(E_i) = p_i$.
        \begin{itemize}
            \item<5-> Note $(\vec{p}, \vec{q}) \in F \implies (U(\vec{p}), \vec{q}) \in F$
        \end{itemize}
        \item<6-> Consider $\theta:\mathcal{D}\rightarrow\mathcal{P}$ defined by $\theta(D) = U_0(D) - U(D)$.
        \begin{itemize}
            \item<7-> Note $\text{im}[\theta] = \mathcal{S}$
        \end{itemize}
        \item<8-> Then, applying $G(\mathcal{E})$-invariance of $F$ implies \[ (U_0(E_1), \cdots, U_0(E_m), q_1 + U_0(D_1) - U(D_1), \cdots, q_n + U_0(D_n) - U(D_n)) \] \[ \in F \] 
    \end{itemize}
\end{frame}

\begin{frame}
    \frametitle{The $\Pi$ Theorem for Relations}
    \alert{\textit{Proof} cont.}
    \begin{itemize}
        \item<2-> Comparing to the definition of $\Phi$, we see that \[ (q_1 - U(D_1), \cdots, q_n - U(D_n)) \in \Phi, \] and
        \item<3-> \[ U(D_i) = U(\sum k_{ij}E_j) = \sum k_{ji}U(E_j) = \sum k_{ij}p_j \qedsymbol \]
    \end{itemize}
\end{frame}

\begin{frame}
    \frametitle{An (Obvious) Proposition}
    \begin{itemize}
        \item<1-> We say a function $f:\avg{E_1}\times\cdots\times\avg{E_m}\times\avg{D_1}\cdots\avg{D_n}\rightarrow\avg{C}$ is $G(\mathcal{E})$-invariant if, for all $\theta \in G(\mathcal{E})$, \[ f(p_1,\cdots,q_n) = c \]\[ \implies f(p_1 + \theta(E_1), \cdots, q_n + \theta(D_n)) = c + \theta(C) \]
        \item<2-> Prop: If $f$ is $G(\mathcal{E})$-invariant, then $C \in \text{Span}\{E_1,\cdots,E_m,D_1,\cdots,D_n\}$
        \item<3-> Proof: Assume not, then consider $\theta$ such that $\theta(E_1) = \cdots = \theta(D_n) = 0$, but $\theta(C) \neq 0$.
        \item<4-> $f$ is clearly not invariant under this map, a contradiction $\qedsymbol$.
    \end{itemize}
\end{frame}

\begin{frame}
    \frametitle{The $\Pi$ Theorem for Functions}
    \alert{\textit{Statement}}
    \begin{itemize}
        \item<2-> Let $f:\avg{E_1}\times\cdots\times\avg{E_m}\times\avg{D_1}\cdots\avg{D_n}\rightarrow\avg{C}$ be a $G(\mathcal{E})$-invariant function.
        \item<3-> Then,
        \begin{itemize}
            \item[1)]<4-> There exists $k_{ij}$ such that $C = \sum k_{n+1, i}E_i$
            \item[2)]<5-> There exists $\phi:\mathcal{S}^n\rightarrow\mathcal{S}$ such that for all $(p_1, \cdots, q_n) \in \avg{E_1}\times\cdots\times\avg{D_n}$, \[ f(\vec{p}, \vec{q}) = \phi(q_1 - \sum k_{1j}p_j, \cdots, q_n - \sum k_{nj}p_j) + \sum k_{n+1,j}p_j \]
        \end{itemize}
    \end{itemize}
\end{frame}

\begin{frame}
    \frametitle{The $\Pi$ Theorem for Functions}
    \alert{\textit{Proof}}
    \begin{itemize}
        \item<2-> The proof of 1) follows immediately from the Obvious Proposition and the linear dependence of the $D_i$ on the $E_i$.
        \item<3-> The proof of 2) follows from the proof the the $\Pi$ Theorem for Relations:
        \begin{itemize}
            \item<4-> We consider the graph of $f$ as an $(n+1)$-ary relation.
            \item<5-> Construct $\Phi \subseteq \mathcal{S}^{n+1}$ as in the proof of the relation version
            \item<6-> $\phi$ is then defined by \[ \Phi = (\lambda_1, \cdots, \lambda_n, \phi(\lambda_1, \cdots, \lambda_n)) \]
        \end{itemize}
    \end{itemize}
\end{frame}

\begin{frame}
    \frametitle{Example: The Harmonic Oscillator}
    \begin{itemize}
        \item<1-> Consider the relation \[ F(m, \hbar, \omega; x, p, H) = H - \frac{p^2}{2m} - \frac{1}{2}m\omega^2x^2 = 0 \]
        \item<2-> In the basis $\mathcal{B} = \{\text{Mass}, \text{Length}, \text{Time}\}$ of $\mathcal{D}$ we have
        \item<3-> \[ [m] = \mqty(1\\0\\0), \; [\hbar] = \mqty(1\\2\\-1), \; [\omega] = \mqty(0\\0\\-1). \]
        \item<4-> These vectors span the space: \[ |\mathcal{M}| = \mqty|1&1&0\\0&2&0\\0&-1&-1| =-2 \neq 0 \]
    \end{itemize}
\end{frame}

\begin{frame}
    \frametitle{Example: The Harmonic Oscillator (cont.)}
    \begin{itemize}
        \item<1-> We can then solve for the dimensions of the variables in terms of the dimensions of the parameters:
        \item<2->
        \begin{align*}
            [x] = \mqty(0\\1\\0); && \mathcal{M}^{-1}[x] = \mqty(-1/2\\1/2\\-1/2) & \implies [x] = \left[\sqrt{\frac{\hbar}{m\omega}}\right] \\
            [p] = \mqty(1\\1\\-1); && \mathcal{M}^{-1}[p] = \mqty(1/2\\1/2\\1/2) & \implies [p] = \left[\sqrt{m\hbar\omega}\right] \\
            [H] = \mqty(1\\2\\-2); && \mathcal{M}^{-1}[H] = \mqty(0\\1\\1) \; & \implies [H] = \left[\hbar\omega\right]
        \end{align*}
    \end{itemize}
\end{frame}

\begin{frame}
    \frametitle{Going Further}
    \begin{itemize}
        \item<1-> Ugh, this has all only been for positive, non-vanishing quantities!
        \begin{itemize}
            \item<2-> But maybe that's ok?
            \item<3-> To what extent can we get away with only using positive quantities? (Mass is ok, but what about position/velocity/momentum etc.?)
            \item<4-> Can this structure be easily extended to accommodate negative and vanishing quantities?
        \end{itemize}
        \item<5-> The vector space description encapsulates the multiplication of quantities, but it seems we still need to put in the addition rules ``by hand''.
        \begin{itemize}
            \item<6-> The Obvious Proposition seems somewhat related to this.
        \end{itemize}
        \item<7-> Another issue: many quantities are best described not by elements of $\mathbb{R}$, but rather by elements of $\mathbb{R}$-\textit{torsors}.
    \end{itemize}
\end{frame}

\begin{frame}
    \frametitle{Bonus: \textit{Torsors}}
    \begin{itemize}
        \item<1-> ``An affine space is like a vector space that forgot its origin; a torsor is like a group that forgot its identity''.
        \item<2-> Technical definition: a torsor (or \textit{principle homogenous space}) for a group $G$ with identity element $e$ is a set $X$ on which $G$ acts such that
        \begin{itemize}
            \item[1)]<3-> $\forall x \in X, \; ex = x$.
            \item[2)]<4-> $\forall x,y\in X, \exists ! \; g\in G$ such that $gx = y$.
        \end{itemize} 
        \item<5-> If $X$ is a \textit{topological group}, we say  that $X$ is a \textit{topological space} and that the action is \textit{continuous}. 
        \item<6-> If $X$ is a \textit{Lie group}, we say  that $X$ is a \textit{smooth manifold} and that the action is \textit{smooth}.
        \item<7-> If $X$ is a \textit{algebraic group}, we say  that $X$ is a \textit{algebraic variety} and that the action is \textit{regular}.  
    \end{itemize}
\end{frame}

\begin{frame}
    \frametitle{Torsors cont.}
    \begin{itemize}
        \item<1-> Examples
        \begin{itemize}
            \item<2-> Energies ($\mathbb{R}$-torsor)
            \item<3-> Phases ($U(1)$-torsor)
            \item<4-> Dates ($\mathbb{Z}$-torsor)
        \end{itemize}
    \end{itemize}
\end{frame}

\begin{frame}
    \frametitle{One More Thing...}
    \begin{itemize}
        \item<1-> What's up with ``Natural units'' ($c = \hbar = 1$)??
        \item<2-> The units of $c$ and $\hbar$ don't span the space of {mass, length, time}.
        \item<3-> Implicityly use base units in which $c = 1$ [length/time], $\hbar = 1$ [mass $\text{length}^2$ / time], and 1eV has the same value as in SI
        \item<4-> We can then treat dimensions not as $\mathbb{Q}^n$, but just ``mass-energy dimension'' $d\in\mathbb{Q}$.
        \item<5-> How to express this with the vector space notation above?
    \end{itemize}
\end{frame}

\begin{frame}
    \frametitle{Oh! Don't Forget about the 2019 SI Redefinition!}
    \begin{itemize}
        \item<1-> The quantities $\Delta \nu_{\text{Cs}}, c, \hbar, e$, and $k_B$ span the dimensional space $\{\text{mass}, \text{length}, \text{time}, \text{temperature}, \text{charge}\}$.
        \item<2-> Therefore, these quantities form a basis of this space.
        \item<3-> This is the foundation of the 2019 SI redefinition.
        \item<4-> All SI base units are defined as a fixed linear transformation of these quantities. 
    \end{itemize}
\end{frame}

\end{document}