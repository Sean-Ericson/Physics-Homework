\documentclass[xcolor={dvipsnames}]{beamer}
\usetheme{Malmoe}
\usecolortheme{spruce}
\setbeamercolor{item}{fg=PineGreen}
\setbeamertemplate{itemize item}[square]
\setbeamertemplate{itemize subitem}[triangle]
\setbeamertemplate{itemize subsubitem}[square]
\setbeamertemplate{itemize/enumerate subbody begin}{\vspace{0.5em}}
\setbeamercolor*{bibliography entry title}{fg=black}
\setbeamercolor*{bibliography entry author}{fg=black}
\setbeamercolor*{bibliography entry location}{fg=black}
\setbeamercolor*{bibliography entry note}{fg=black}
% and kill the abominable icon
\setbeamertemplate{bibliography item}{}

\usepackage{natbib}
\usepackage{amsmath,amsthm,amssymb}
\usepackage{mathtools}
\usepackage{mathrsfs}
\usepackage{bm}
\usepackage{physics}
\usepackage{slashed}
\usepackage{graphicx}
\usepackage{caption}

\newcommand{\magsq}[1]{\big|#1\big|^2}

\captionsetup[figure]{labelformat=empty}
\graphicspath{ {./images/} }

\let\olditemize=\itemize 
\let\endolditemize=\enditemize 
\renewenvironment{itemize}{\olditemize \itemsep=.5em }{\endolditemize}

\title{All About Spinors}
\subtitle{...on \textit{flat} spacetime}
\author{Sean Ericson}
\institute{UO}
\date{Theory meeting, June 27, 2024}
\titlegraphic{\includegraphics[scale=0.6]{seal.jpg}}

\begin{document}
\frame{\titlepage}

\begin{frame}{Special and General Covariance}
    \alert{Some Philisophical Motivation}
    \begin{itemize}
        \item<2-> \textit{Special covariance}: For any two families of inertial observes $O$ and $O'$ related by an isometry, any set of physical measurements observable by $O$ is observable by $O'$
        \begin{itemize}
            \item<3-> Expresses invariance of physical laws under isometries
            \item<4-> Implies an action of the group of isometries on the space of physical states $\tilde{\phi}_g:\mathcal{S}\rightarrow\mathcal{S}$
        \end{itemize} 
        \item<5-> \textit{General covariance}: For a physical entity described by a tensor field $T_{\quad b\dots}^{a\dots}$, the equations governing the field should be of the form $f(T, \partial T, \cdots, g_{ab}, \partial g_{ab}, \cdots)$
        \begin{itemize}
            \item<6-> Expresses the invariance of physical laws under diffeomorphisms
            \item<7-> G.C. $\implies$ S.C.
        \end{itemize}
        \item<8-> Special/General covariance \textrightarrow\; Special/General relativity
    \end{itemize}
\end{frame}

\begin{frame}{Minkowski Space and Quantum Theory}
    \begin{itemize}
        \item<2-> Consider $(\mathbb{R}^4, \eta_{ab})$, $\mathcal{S} = \{\psi \in \mathcal{H}\;:\; \magsq{\psi} = 1\}/\sim$
        \begin{itemize}
            \item<3-> $\psi \sim \psi' \;\iff\; \psi = e^{i\alpha}\psi' $
        \end{itemize}  
        \item<4-> Isometry group: $G = ISO(3, 1)^+$
        \item<5-> Associate with $\tilde{\phi}_g$ a map $U_g: \mathcal{H}\rightarrow\mathcal{H}$ which preserves $\sim$
        \begin{itemize}
            \item<6-> $U_g$ can be rephased to be (anti-)unitary
        \end{itemize}
        \item<7-> Composition: $\tilde{\phi}_{g_1}\circ\tilde{\phi}_{g_2} = \tilde{\phi}_{g_1g_2} \;\implies\; U_{g_1}U_{g_2} = e^{i\theta}U_{g_1g_2}$
        \begin{itemize}
            \item<8-> Wigner ('39): can set phases so $\theta = n\pi$, i.e. $U_{g_1}U_{g_2} = \pm U_{g_1g_2}$
        \end{itemize}
        \item<9-> $\mathcal{H}$ a rep. space for a unitary rep. (up to sign) of $ISO(3, 1)^+$!
        \item<10-> Bargmann ('54): reps up to sign are \textit{exactly} the true reps of the universal cover
    \end{itemize}
\end{frame}

\begin{frame}{Universal Covering Spaces}
    \begin{itemize}
        \item<2-> The universal cover $\mathcal{U}(M)$ of a topological space $M$ is a simply connected space which covers $M$
        \item<3-> Construction:
        \begin{itemize}
            \item<4-> Cut $M$ such that it becomes simply connected with boundary
            \item<5-> Glue together copies to eliminate the boundaries
        \end{itemize}
        \item<6-> Lie group structure of $M$ is naturally lifted to $\mathcal{U}(M)$
        \item<7-> Fundamental group of $ISO(3, 1)^+$ is $\mathbb{Z}_2$ \textrightarrow\; double cover
        \item<8-> In fact, $\mathcal{U}(ISO(3, 1)^+) \cong ISL(2, \mathbb{C})$
    \end{itemize}
\end{frame}

\begin{frame}{Spinors, Spinorial Tensors, and Spinor Space}
    \begin{itemize}
        \item<2-> Let $W \cong \mathbb{C}^2$
        \begin{itemize}
            \item<3-> \textit{Dual space} $W^*$: linear maps $\lambda_A : W \rightarrow \mathbb{C}$
            \item<4-> \textit{Conjugate dual space} $\overline{W}^*$: anti-linear maps $\lambda_{A'} : W \rightarrow \mathbb{C}$ 
            \item<5-> \textit{Conjugate space} $\overline{W}$: the dual space of $\overline{W}^*$ 
        \end{itemize}
        \item<6-> Tensor of type $(k,l;k',l')$:
        \[ T_{B_1\dots B_l B_1'\dots B_{l'}'}^{A_1\dots A_k A_1'\dots A_{k'}'}: \left(W^*\right)^k \times \left(W\right)^l \times \left(\overline{W}^*\right)^{k'} \times \left(\overline{W}\right)^{l'} \rightarrow \mathbb{C} \]
        \item<7-> Space of type (0,2;0,0) antisymmetric tensors is 1-dimensional
        \item<8-> Spinor space: $(W, \epsilon_{AB})$
        \begin{itemize}
            \item<9-> $\lambda^A \in W$ is called a \textit{spinor}
            \item<10-> Tensors over $W$ are called \textit{spinoral tensors}
        \end{itemize}
    \end{itemize}
\end{frame}

\begin{frame}{Spinor Conventions}
    \begin{itemize}
        \item<2-> (Un)primed index order irrelevant: $T_{\qquad C}^{AD'B} \leftrightarrow T_{\quad C}^{AB\;\;D'}$ 
        \item<3-> Conjugation maps $(k,l;k',l')$ tensors to $(k',l';k,l)$ tensors: $$T_{\;\;BC}^A \leftrightarrow \overline{T}_{\;\;B'C'}^{A'}$$
        \item<4-> $\epsilon^{AB}$,$\epsilon_{AB}$ raise/lower unprimed indices; $\overline{\epsilon}_{A'B'}$ for primed
        \item<5-> Contraction occurs over \textit{first} index of $\epsilon$: \[ \phi_A = \epsilon_{BA}\phi^B = -\epsilon_{AB}\phi^{B} \]
        \begin{itemize}
            \item<6-> $\implies \phi_A \phi^A = 0$
        \end{itemize}
        \item<7-> $\delta_{\;\;B}^A = \mathbb{I}_W$ differs by a sign from $\delta_C^{\;\;D} = \mathbb{I}_{W^*}$
        \begin{itemize}
            \item<8-> \textrightarrow\; use $\epsilon_{\;\;B}^A$, $\epsilon_C^{\;\;D}$ and their conjugates to avoid confusion
        \end{itemize}
    \end{itemize}
\end{frame}

\begin{frame}{$SL(2, \mathbb{C})$ and $SO(3, 1)^+$}
    \begin{itemize}
        \item<2-> Let $L_{\;\;B}^A : W\rightarrow W$ be a linear transformation
        \begin{itemize}
            \item<3-> $\det(L) \coloneqq \frac{1}{2}\epsilon_{AB}\epsilon_{CD}L_{\;\;C}^AL_{\;\;D}^B$
        \end{itemize}
        \item<4-> $SL(2, \mathbb{C})$ is simply all $L$ with $\det(L) = 1$
        \begin{itemize}
            \item<5-> Polar decomp: $L = UH$ \textrightarrow\; 6 real d.o.f.
            \item<6-> Simply connected Lie group $\cong S^3\times\mathbb{R}^3$
            \item<7-> $\det(L)=1 \; \iff L_{\;\;C}^A L_{\;\;D}^B \epsilon_{AB} = \epsilon_{CD}$
        \end{itemize}
    \end{itemize}
\end{frame}

\begin{frame}{$SL(2, \mathbb{C})$ and $SO(3, 1)^+$}
    \begin{itemize}
        \item<1-> Tensors $\phi^{AA'} \in W_{1,0;1,0}$ comprise a $\mathbb{C}^4$ vector space
        \begin{itemize}
            \item<2-> Let $\{o^A, \iota^A \}$ be a basis for $W$ with $o_A\iota^A = 1$
            \item<3-> A basis for $W_{1,0;1,0}$ can be given by
            \begin{align*}
                t^{AA'} &= \frac{1}{\sqrt{2}}\left(o^A\overline{o}^{A'} + \iota^A\overline{\iota}^{A'}\right) &&= \frac{1}{\sqrt{2}}\mqty(1&0\\0&1) \\
                x^{AA'} &= \frac{1}{\sqrt{2}}\left(o^A\overline{\iota}^{A'} + \iota^A\overline{o}^{A'}\right) &&= \frac{1}{\sqrt{2}}\mqty(0&1\\1&0) \\
                y^{AA'} &= \frac{i}{\sqrt{2}}\left(o^A\overline{\iota}^{A'} - \iota^A\overline{o}^{A'}\right) &&= \frac{1}{\sqrt{2}}\mqty(0&i\\-i&0) \\
                z^{AA'} &= \frac{1}{\sqrt{2}}\left(o^A\overline{o}^{A'} - \iota^A\overline{\iota}^{A'}\right) &&= \frac{1}{\sqrt{2}}\mqty(1&0\\0&-1)
            \end{align*}
        \end{itemize}
        \vspace{-1em}
        \item<4-> Under conjugation, $\overline{W}_{1,0;1,0} = W_{1,0;1,0}$
        \begin{itemize}
            \item<5-> $\phi^{AA'} \in W_{1,0;1,0}$ s.t. $\overline{\phi}^{AA'} = \phi^{AA'}$ are called \textit{real}
        \end{itemize}
    \end{itemize}
\end{frame}

\begin{frame}{$SL(2, \mathbb{C})$ and $SO(3, 1)^+$}
    \begin{itemize}
        \item<1-> $\{t^{AA'}, x^{AA'}, y^{AA'}, z^{AA'}\}$ defined above are clearly real
        \item<2-> They span a 4-real dimensional space $V \subset W_{1,0;1,0}$
        \item<3-> Define $g:V\times V \rightarrow \mathbb{R}$ by $g_{AA'BB'} \coloneqq \epsilon_{AB}\overline{\epsilon}_{A'B'}$
        \begin{itemize}
            \item<4-> $g$ is nondegenerate with signature $(+, -, -, -)$; a Lorentz metric!
        \end{itemize}
        \item<5-> Define $\lambda: V \rightarrow V$ by $\lambda_{\quad BB'}^{AA'} \coloneqq L_{\;\;B}^A \overline{L}_{\;\;B'}^{A'}$
        \begin{itemize}
            \item<6-> Automatically, $\lambda_{\quad CC'}^{AA'}\lambda_{\quad DD'}^{BB'} g_{AA'BB'} = g_{CC'DD'}$
            \item<7-> But this means $\lambda \in O(3, 1)$ (in fact, $SO(3, 1)^+$)!!
            \item<8-> $L_1$, $L_2$ \textrightarrow\; $\lambda$ $\implies L_1 = \pm L_2$ (a double cover)
        \end{itemize}
        \item<9-> Let $\{t^a, x^a, y^a, z^a \}$ be a basis for $\mathbb{R}^{3,1}$
        \item<10-> Define $\sigma_{\;AA'}^a \coloneqq t^at_{AA'} - x^ax_{AA'} - y^ay_{AA'} - z^az_{AA'}$
        \begin{itemize}
            \item<11-> $\sigma$ is an isomorphism between $\Re[W_{1,0;1,0}]$ and $\mathbb{R}^{3,1}$
        \end{itemize}
    \end{itemize}
\end{frame}

\begin{frame}{Spinors and Null Vectors}
    \begin{itemize}
        \item<2-> Let $\psi^A \in W$, then $\psi^A\bar{\psi}^{A'} \eqqcolon k^{AA'} \in \Re[W_{1,0;1,0}]$
        \begin{itemize}
            \item<3-> $k_{AA'}k^{AA'} = g_{AA'BB'}k^{AA'}k^{BB'} = \epsilon_{AB}\overline{\epsilon}_{A'B'}\psi^{A}\bar{\psi}^{A'}\psi^{B}\bar{\psi}^{B'} = 0$
            \item<4-> $k^{AA'}$ is thus a null vector ($\psi^A$ ``square root'' of a null vector\textinterrobang) 
        \end{itemize}
        \item<5-> Let $\psi^A, \phi^A \in W$, then $\psi_A\overline{\psi}_{A'}\phi^{A}\bar{\phi}^{A'} = \magsq{\psi_A\phi^A} > 0$
        \begin{itemize}
            \item<6-> $k_\psi^{AA'}$, $k_\phi^{AA'}$ on the same side of the light cone: future direction
            \item<7-> $\Re[W_{1,0;1,0}]$ has a natural time orientation
        \end{itemize}
        \item<8-> Def. $\epsilon_{AA'BB'CC'DD'} \coloneqq \epsilon_{AB}\epsilon_{CD}\overline{\epsilon}_{A'C'}\overline{\epsilon}_{B'D'} - \epsilon_{AC}\epsilon_{BD}\overline{\epsilon}_{A'B'}\overline{\epsilon}_{C'D'}$
        \begin{itemize}
            \item<9-> $\Re[W_{1,0;1,0}]$ has a natural orientation
        \end{itemize}
        \item<10-> \textit{Null flag}: $F^{AA'BB'} \coloneqq \psi^A\psi^B\overline{\epsilon}^{A'B'} + \bar{\psi}^{A'}\bar{\psi}^{B'}\epsilon^{AB}$
        \begin{itemize}
            \item<11-> $F^{AA'BB'} = -F^{BB'AA'}, \; F_{AA'BB'}F^{AA'BB'} = F_{AA'BB'}\psi^{B}\bar{\psi}^{B'} = 0$
            \item<12-> $\implies F^{AA'BB'} = k^{AA'}m^{BB'} - k^{BB'}m^{AA'}$ for some $m^{AA'}$
            \item<13-> $\psi, \psi' \rightarrow F \iff \psi = \pm \psi'$
        \end{itemize} 
    \end{itemize}    
\end{frame}

\begin{frame}{A Couple Neat Identites}
    \begin{itemize}
        \item<2-> Let $T_{ab}$ ($T_{AA'BB'}$) be a tensor on $\mathbb{R}^{3,1}$ ($\Re[W_{0,1;0,1}]$)
        \item<3-> Antisymmetrization:
        \begin{itemize}
            \item<4-> $T_{[ab]} = T_{(AB)[A'B']} + T_{[AB](A'B')} = \phi_{AB}\overline{\epsilon}_{A'B'} + \overline{\phi}_{A'B'}\epsilon_{AB}\qquad$ with $\phi_{AB} = \frac{1}{2}T_{(AB)A'}^{\qquad\; A'}$ symmetric
        \end{itemize}
        \item<5-> Symmetrization:
        \begin{itemize}
            \item<6-> $T_{(ab)} = T_{(AB)(A'B')} + T_{[AB][A'B']} = T_{(AB)(A'B')} + \frac{1}{4}\epsilon_{AB}\overline{\epsilon}_{A'B'}T$ where $T = T_{A\;\;A}^{\;\;A\;\;A} = T_aT^a$
            \item<7-> Note also that $T_{[AB]A'}^{\qquad\;A'} = \frac{1}{2}\epsilon_{AB}^T$ 
        \end{itemize}
        \item<8-> $\partial_{AA'}\partial_B^{\;\;A'} = \frac{1}{2}\epsilon_{AB}\Box, $ where $\Box = \partial_{AA'}\partial^{AA'} $
    \end{itemize}
\end{frame}

\begin{frame}{The Universal Enveloping Algebra}
    \begin{itemize}
        \item<2-> Given a Lie algebra $\mathfrak{g}$, the \textit{universal enveloping algebra} $\mathscr{U}(\mathfrak{g})$ is the unique unital associative algebra whose representations correspond exactly to the representations of $\mathfrak{g}$
        \item<3-> Construction
        \begin{itemize}
            \item<4-> Form the \textit{Tensor algebra} $T(\mathfrak{g}) \coloneqq \mathbb{C} \oplus \mathfrak{g} \oplus \left(\mathfrak{g}\otimes\mathfrak{g}\right) \cdots$
            \item<5-> Recursively lift $[,]$ from $\mathfrak{g}$ to $T(\mathfrak{g})$
            \item<6-> Fully lifted, $[,]$ obeys Leibniz's law: $T(\mathfrak{g})$ is a \textit{poisson} algebra
            \item<7-> $\mathscr{U}(\mathfrak{g})$ is what remains after ``modding out'' the poisson structure, i.e. $\mathscr{U}(\mathfrak{g}) = T(\mathfrak{g}) / \sim$ where the equivalence relation is given by $[a,b] = a\otimes b - b\otimes a$ 
        \end{itemize}
        \item<8-> Casimir elements
        \begin{itemize}
            \item<9-> \textit{Center} $Z(\mathscr{U}(\mathfrak{g}))$: all elements that commute with all of $\mathscr{U}(\mathfrak{g})$
            \item<10-> \textit{Casimir elements} form a basis of $Z(\mathscr{U}(\mathfrak{g}))$
            \item<11-> Casimir representatives proportional to identity
        \end{itemize}
    \end{itemize}
\end{frame}

\begin{frame}{Representations and Hilbert Space}
    \begin{itemize}
        \item<2-> $\mathscr{U}(ISL(2, \mathbb{C}))$ has two independent Casimir elements: $P^2$ and $S^2$ with eigenvalues $m^2$ and $s(s+1)$, respectively.
        \item<3-> Classes:
        \begin{enumerate}
            \item<4-> $m^2 > 0$
            \item<5-> $m^2 = 0$; nontrivial translations
            \begin{enumerate}
                \item<6-> Helicity parameterization: $s = 0, \pm \frac{1}{2}, \pm 1, \pm\frac{3}{2}\dots$
                \item<7-> ``Continuous spin''
            \end{enumerate}
            \item<8-> $m^2 = 0$; trivial translations
            \item<9-> $m^2 < 0$ (tachyons!)
        \end{enumerate}
        \item<10-> Physically relevant cases are 1 and 2.1
    \end{itemize}    
\end{frame}

\begin{frame}{Representations and Hilbert Space}
    \begin{itemize}
        \item<1-> Let $n = 2s$. Class 1 reps selected by \[ \left(\Box + m^2\right)\phi^{A_1\dots A_n} = 0 \]
        \item<2-> or, equivalently
        \begin{align*}
            \partial_{AA'}\phi^{A_1\dots A_n} &= \frac{m}{\sqrt{2}}\xi_{A_1'}^{\;\;\;A_2\dots A_n} \\
            \partial_{AA'}\xi_{A_1'}^{\;\;\;A_2\dots A_n} &= -\frac{m}{\sqrt{2}}\phi^{A_1\dots A_n}
        \end{align*}
        \item<3-> For $s = 1/2$, the pair $(\phi^A, \xi_{A'})$ is known as a \textit{Dirac} spinor, and the above equations are just the Dirac equation
    \end{itemize}
\end{frame}

\begin{frame}{Representations and Hilbert Space}
    \begin{itemize}
        \item<1-> For class 2.1, previous equations do not select irreducible reps
        \item<2-> Irreps for class 2.1 given by \[ \partial_{A_1'A_1}\phi^{A_1\dots A_n} = 0 \]
        \begin{itemize}
            \item<3-> $s = \frac{1}{2}$: Weyl neutrino equation
            \item<4-> $s = 1$: Maxwell's equations
            \item<5-> $s = 2$: Linearized GR
        \end{itemize}
    \end{itemize}
\end{frame}

\begin{frame}{Representations and Hilbert Space}
    \begin{itemize}
        \item<1-> We need an inner product for this Hilbert space
        \item<2-> Define the (conserved) \textit{particle current vector}:
        \begin{align*}
            j^{AA'}(\phi, \psi) &\coloneqq (-i)^{n-1}\big( \overline{\phi}^{A'A_2'\dots A_n'}\partial_{A_2'A_2}\cdots\partial_{A_n'A_n}\psi^{AA_2\dots A_n} \\
            &\qquad\qquad\quad + \overline{\xi}^{AA_2'\dots A_n}\partial_{A_2'A_2}\cdots\partial_{A_n'A_n}\zeta^{A'A_2\dots A_n} \big) 
        \end{align*}
        where $\xi$ and $\zeta$ are the auxiliary fields corresponding to $\phi$ and $\psi$, respectively
        \item<3-> Define an inner product by integrating the normal component of the particle current of a Cauchy surface $\Sigma$:
        \[ \left<\phi, \psi\right> \coloneqq \int_\Sigma j^{AA'}n_{AA'}\dd V \]  
    \end{itemize}
\end{frame}

\begin{frame}{Summary}
    \begin{itemize}
        \item<2-> The natural action of $ISL(2, \mathbb{C})$ on spinoral tensor fields $\phi^{A_1\dots A_n}$ gives rise to all known physical fields
        \begin{itemize}
            \item<3-> For even $n$ (bosons), the representations are true reps of the Poincar\'e group (spinors not actually required)
            \item<4-> For odd $n$ (fermions), the reps are only reps up to sign, and spinors are necessary to describe them
        \end{itemize}  
    \end{itemize}    
\end{frame}

\begin{frame}{References}
    \nocite{wald:1984}
    \nocite{Wikipedia:2024}
    \bibliographystyle{abbrv}
    \bibliography{refs}
\end{frame}

\end{document}