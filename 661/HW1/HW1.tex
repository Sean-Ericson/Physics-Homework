\documentclass[12pt]{article}

\usepackage[margin=1in]{geometry}
\usepackage{amsmath,amsthm,amssymb}
\usepackage{mathtools}
\usepackage{mathrsfs}
\usepackage{enumitem}
\usepackage{physics}
\usepackage{tensor}

\usepackage{tikz}
\usetikzlibrary{calc,decorations.markings,patterns}

\newcommand{\magsq}[1]{\big|#1\big|^2}
\newcommand{\avg}[1]{\left<#1\right>}
\newcommand{\fullint}{\int_{-\infty}^\infty}
\newcommand{\fullintd}[1]{\fullint\dd#1\:}
\newcommand{\cint}[2]{\int_{#1}^{#2}}
\newcommand{\cintd}[3]{\cint{#1}{#2}\dd#3\:}
\newcommand{\boostXY}{\mqty(\gamma & -\gamma\beta_x & -\gamma\beta_y & 0 \\ -\gamma\beta_x & 1 + (\gamma - 1)\frac{\beta_x^2}{\abs{\beta}^2} & (\gamma - 1)\frac{\beta_x\beta_y}{\abs{\beta}^2} & 0 \\ -\gamma\beta_y & (\gamma - 1)\frac{\beta_x\beta_y}{\abs{\beta}^2} & 1 + (\gamma - 1)\frac{\beta_y^2}{\abs{\beta}^2} & 0 \\ 0 & 0 & 0 & 1)}
\newcommand{\metric}{\mqty(1 & 0 & 0 & 0 \\ 0 & -1 & 0 & 0 \\ 0 & 0 & -1 & 0 \\ 0 & 0 & 0 & -1)}

\begin{document}

\title{Homework 1}
\author{Sean Ericson \\ Phys 661}
\maketitle

\section*{Problem 1}
\begin{enumerate}[label=(\alph*)]
    \item \[ \frac{\hbar c}{r_J} \approx 2.82\times 10^{-21}\;\text{MeV} \]
    \item \[ \frac{\hbar c}{\lambda_W} \approx 2.45\times 10^{-16}\;\text{cm} \]
    \item \[ \frac{\hbar c}{\sqrt{s}} \approx 9.9\times 10^{-17}\;\text{cm} \]
\end{enumerate}


\section*{Problem 2}
Conservation of energy implies
\begin{equation*} \tag{2.1}
    E + m_e = E' + E_e',
\end{equation*}
    
while conservation of momentum implies
\begin{align*}
    E' \sin\theta - \abs{p_e'}\sin\theta' &= 0 \\
    E' \cos\theta - \abs{p_e'}\cos\theta' &= E. \tag{2.2}
\end{align*}
Now we note that the final electron energy is given by
\begin{equation*} \tag{2.3}
    (E_e')^2 = \abs{p_e'}^2 + m_e^2.
\end{equation*}
Plugging (2.3) into (2.1), then solving the result along with (2.3) for $\abs{p_e'}^2$, we find
\begin{align*}
    \abs{p_e'}^2 &= (E - E' + m_e)^2 - m_e^2 \\
    \abs{p_e'}^2 &= E^2 + (E')^2 - 2EE'\cos\theta \tag{2.4}.
\end{align*}
Equating the two right hand sides of (2.4) gives
\begin{align*}
    &&E^2 + (E')^2 - 2EE'\cos\theta &= E^2 + (E')^2 + m_e^2 + 2(E-E')m_e - 2EE' - m_e^2 \\
    \implies&& 2(E-E')m_e - 2EE' &= -2EE'\cos\theta \\
    \implies&& (\frac{1}{E'} - \frac{1}{E})m_e &= 1 - \cos\theta
\end{align*}
Finally, using the photon wavelength/energy relation $E = \frac{2\pi}{\lambda}$, we arrive at the desired equation for wavelength shift:
\[ (\frac{\lambda'}{2\pi} - \frac{\lambda}{2\pi})m_e = (1-cos\theta) \implies \boxed{\lambda' - \lambda = \frac{2\pi}{m_e}(1-\cos\theta)} \]

\section*{Problem 3}
Given that
\begin{equation*} \tag{3.1}
    (p')^\mu = \Lambda_{\;\nu}^\mu p^\nu 
\end{equation*}
we have that 
\begin{align*}
    (p')_\mu(p')^\mu &= \Lambda_{\mu\nu}p^\nu \Lambda_{\;\rho}^\mu p^\rho \\
    &= \eta_{\mu\alpha}\Lambda_{\;\nu}^\alpha\Lambda_{\;\rho}^\mu p^\nu p^\rho \tag{3.2}
\end{align*}
Now, if
\begin{equation*} \tag{3.3}
    \eta_{\mu\alpha}\Lambda_{\;\nu}^\alpha\Lambda_{\;\rho}^\mu = \eta_{\nu\rho},
\end{equation*}
then the right hand side of (3.2) becomes
\begin{equation*} \tag{3.4}
    \eta_{\nu\rho} p^\nu p^\rho = p_\nu p^\nu, 
\end{equation*}
which is the result we wish to show. Now, (3.4) is equivalent to $\Lambda^\intercal \eta \Lambda$. Calculating $\eta \Lambda$, we find
\begin{equation*}
    \eta \boostXY = \mqty(\gamma & -\gamma\beta_x & -\gamma\beta_y & 0 \\ \gamma\beta_x & -1 - (\gamma - 1)\frac{\beta_x^2}{\abs{\beta}^2} & (1 - \gamma)\frac{\beta_x\beta_y}{\abs{\beta}^2} & 0 \\ \gamma\beta_y & (1- \gamma)\frac{\beta_x\beta_y}{\abs{\beta}^2} & -1 - (\gamma - 1)\frac{\beta_y^2}{\abs{\beta}^2} & 0 \\ 0 & 0 & 0 & -1).
\end{equation*}
Finally, multiplying on the left by $\Lambda^\intercal = \Lambda$ and simplifying, we arrive back at the metric:
\[ \Lambda^\intercal \eta \Lambda = \metric = \eta \].
Thus, (3.3) is satisfied and therefore $(p')_\mu(p')^\mu = p_\mu p^\mu$.

\section*{Problem 4}
In the center of mass-energy frame, we have 
\begin{equation*} \tag{4.1}
    (M, 0, 0, 0) \rightarrow 
    \begin{cases*}
        (\frac{M}{2}, \frac{1}{4}M^2 - m^2, 0, 0) \\
        (\frac{M}{2}, m^2 - \frac{1}{4}M^2 0, 0),    
    \end{cases*}
\end{equation*}
while in the lab frame we have
\begin{equation*} \tag{4.2}
    (\gamma M, 0, 0, \gamma\beta) \rightarrow 
    \begin{cases*}
        (\sqrt{m^2 + \gamma^2\abs{\beta_\text{Lab}'}}, \gamma\abs{\beta_\text{Lab}'}\sin\frac{\theta}{2}, 0, \gamma\abs{\beta_\text{Lab}'}\cos\frac{\theta}{2}) \\
        (\sqrt{m^2 + \gamma^2\abs{\beta_\text{Lab}'}}, -\gamma\abs{\beta_\text{Lab}'}\sin\frac{\theta}{2}, 0, \gamma\abs{\beta_\text{Lab}'}\cos\frac{\theta}{2}),
    \end{cases*}
\end{equation*}
where $\theta$ is the angle between the final state particles in the lab frame. Boosting from the CoM/E frame to labe frame (i.e. boosting by $-\beta \hat{z}$), we find
\begin{equation*} \tag{4.3}
    \begin{rcases*}
        (\frac{M}{2}, \frac{1}{4}M^2 - m^2, 0, 0) \\
        (\frac{M}{2}, m^2 - \frac{1}{4}M^2 0, 0)  
    \end{rcases*} 
    \rightarrow \mqty{(\frac{1}{2}\gamma M, \frac{1}{4}M^2 - m^2, 0, -\gamma\beta m) \\ (\frac{1}{2}\gamma M, m^2 - \frac{1}{4}M^2, 0, -\gamma\beta m)}.
\end{equation*}
By conservation of energy, we have that 
\begin{equation*} \tag{4.4}
    \sqrt{m^2 + \gamma^2\abs{\beta_\text{Lab}'}^2} = \frac{1}{2}\gamma M \implies \abs{\beta_\text{Lab}'}^2 = \frac{1}{4}M^2 - (\frac{m}{\gamma})^2
\end{equation*}
Now, comparing the $z$ components of (4.2) and (4.3), we see that
\begin{align*} \tag{4.5}
    \gamma \abs{\beta_\text{Lab}} \cos\frac{\theta}{2} = -\gamma\beta \implies \boxed{\cos\frac{\theta}{2} = \frac{-\beta}{\frac{1}{4}M^2 - (\frac{m}{\gamma})^2}} 
\end{align*}
Oops, this can't be right...

\end{document}